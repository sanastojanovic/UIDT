%
\begin{isabellebody}%
\setisabellecontext{MyTheory}%
%
\isadelimtheory
%
\endisadelimtheory
%
\isatagtheory
%
\endisatagtheory
{\isafoldtheory}%
%
\isadelimtheory
%
\endisadelimtheory
%
\begin{exercise}[subtitle=Algebra skupova]
%
\begin{isamarkuptext}%
Diskutovati o sledećim termovima:%
\end{isamarkuptext}\isamarkuptrue%
\isacommand{term}\isamarkupfalse%
\ {\isachardoublequoteopen}{\isacharbraceleft}{\kern0pt}{\isadigit{1}}{\isacharcomma}{\kern0pt}\ {\isadigit{2}}{\isacharcomma}{\kern0pt}\ {\isadigit{3}}{\isacharbraceright}{\kern0pt}{\isachardoublequoteclose}\isanewline
\isacommand{term}\isamarkupfalse%
\ {\isachardoublequoteopen}{\isacharbraceleft}{\kern0pt}{\isadigit{1}}{\isacharcolon}{\kern0pt}{\isacharcolon}{\kern0pt}nat{\isacharcomma}{\kern0pt}\ {\isadigit{2}}{\isacharcomma}{\kern0pt}\ {\isadigit{3}}{\isacharbraceright}{\kern0pt}{\isachardoublequoteclose}\isanewline
\isacommand{term}\isamarkupfalse%
\ {\isachardoublequoteopen}{\isacharparenleft}{\kern0pt}{\isasymin}{\isacharparenright}{\kern0pt}{\isachardoublequoteclose}\isanewline
\isacommand{term}\isamarkupfalse%
\ {\isachardoublequoteopen}{\isacharparenleft}{\kern0pt}{\isasyminter}{\isacharparenright}{\kern0pt}{\isachardoublequoteclose}\isanewline
\isacommand{term}\isamarkupfalse%
\ {\isachardoublequoteopen}{\isacharparenleft}{\kern0pt}{\isasymunion}{\isacharparenright}{\kern0pt}\ A{\isachardoublequoteclose}\isanewline
\isacommand{term}\isamarkupfalse%
\ {\isachardoublequoteopen}{\isacharparenleft}{\kern0pt}A{\isacharcolon}{\kern0pt}{\isacharcolon}{\kern0pt}{\isacharprime}{\kern0pt}a\ set{\isacharparenright}{\kern0pt}\ {\isacharminus}{\kern0pt}\ B{\isachardoublequoteclose}%
\end{exercise}
%
\begin{exercise}[subtitle=Isar dokazi]
%
\begin{isamarkuptext}%
Pokazati sledeća tvrđenja o skupovima pomoću jezika Isar.%
\end{isamarkuptext}\isamarkuptrue%
%
\begin{isamarkuptext}%
\isa{Napomena}: Dozvoljeno je korišćenje samo \isa{simp} metode za
                  dokazivanje pojedinačnih koraka.%
\end{isamarkuptext}\isamarkuptrue%
\isacommand{lemma}\isamarkupfalse%
\ {\isachardoublequoteopen}{\isacharminus}{\kern0pt}\ {\isacharparenleft}{\kern0pt}A\ {\isasymunion}\ B{\isacharparenright}{\kern0pt}\ {\isacharequal}{\kern0pt}\ {\isacharminus}{\kern0pt}\ A\ {\isasyminter}\ {\isacharminus}{\kern0pt}\ B{\isachardoublequoteclose}\isanewline
\ \ %
\isadelimproof
%
\endisadelimproof
%
\isatagproof
%
\endisatagproof
{\isafoldproof}%
%
\isadelimproof
%
\endisadelimproof
%
\begin{isamarkuptext}%
\isa{Savet}: Iskoristiti \isa{find{\isacharunderscore}{\kern0pt}theorems\ {\isacharunderscore}{\kern0pt}\ {\isasymor}\ {\isacharunderscore}{\kern0pt}\ {\isasymlongleftrightarrow}\ {\isacharunderscore}{\kern0pt}\ {\isasymor}\ {\isacharunderscore}{\kern0pt}} 
               za pronalaženje odgovarajuće teoreme.%
\end{isamarkuptext}\isamarkuptrue%
\isacommand{lemma}\isamarkupfalse%
\ {\isachardoublequoteopen}A\ {\isasymunion}\ B\ {\isacharequal}{\kern0pt}\ B\ {\isasymunion}\ A{\isachardoublequoteclose}\isanewline
%
\isadelimproof
%
\endisadelimproof
%
\isatagproof
\isacommand{proof}\isamarkupfalse%
\isanewline
\ \ \isacommand{show}\isamarkupfalse%
\ {\isachardoublequoteopen}A\ {\isasymunion}\ B\ {\isasymsubseteq}\ B\ {\isasymunion}\ A{\isachardoublequoteclose}\isanewline
\ \ \isacommand{proof}\isamarkupfalse%
\isanewline
\ \ \ \ \isacommand{fix}\isamarkupfalse%
\ x\isanewline
\ \ \ \ \isacommand{assume}\isamarkupfalse%
\ {\isachardoublequoteopen}x\ {\isasymin}\ A\ {\isasymunion}\ B{\isachardoublequoteclose}\isanewline
\ \ \ \ \isacommand{show}\isamarkupfalse%
\ {\isachardoublequoteopen}x\ {\isasymin}\ B\ {\isasymunion}\ A{\isachardoublequoteclose}\isanewline
\ \ \ \ \ \ \ \isacommand{qed}\isamarkupfalse%
\isanewline
\isacommand{next}\isamarkupfalse%
\isanewline
\ \ \isacommand{show}\isamarkupfalse%
\ {\isachardoublequoteopen}B\ {\isasymunion}\ A\ {\isasymsubseteq}\ A\ {\isasymunion}\ B{\isachardoublequoteclose}\isanewline
\ \ \ \ \isacommand{qed}\isamarkupfalse%
%
\endisatagproof
{\isafoldproof}%
%
\isadelimproof
%
\endisadelimproof
%
\begin{isamarkuptext}%
\isa{Savet}: Iskoristiti aksiomu isključenja trećeg \isa{A\ {\isasymor}\ {\isasymnot}A}
               u kontekstu operacije pripadanja \isa{{\isacharparenleft}{\kern0pt}{\isasymin}{\isacharparenright}{\kern0pt}\ {\isacharcolon}{\kern0pt}{\isacharcolon}{\kern0pt}\ {\isacharprime}{\kern0pt}a\ {\isasymRightarrow}\ {\isacharprime}{\kern0pt}a\ set\ {\isasymRightarrow}\ bool}.%
\end{isamarkuptext}\isamarkuptrue%
\isacommand{lemma}\isamarkupfalse%
\ {\isachardoublequoteopen}A\ {\isasymunion}\ {\isacharparenleft}{\kern0pt}B\ {\isasyminter}\ C{\isacharparenright}{\kern0pt}\ {\isacharequal}{\kern0pt}\ {\isacharparenleft}{\kern0pt}A\ {\isasymunion}\ B{\isacharparenright}{\kern0pt}\ {\isasyminter}\ {\isacharparenleft}{\kern0pt}A\ {\isasymunion}\ C{\isacharparenright}{\kern0pt}{\isachardoublequoteclose}\ {\isacharparenleft}{\kern0pt}\isakeyword{is}\ {\isachardoublequoteopen}{\isacharquery}{\kern0pt}left\ {\isacharequal}{\kern0pt}\ {\isacharquery}{\kern0pt}right{\isachardoublequoteclose}{\isacharparenright}{\kern0pt}\isanewline
%
\isadelimproof
%
\endisadelimproof
%
\isatagproof
\isacommand{proof}\isamarkupfalse%
\isanewline
\ \ \isacommand{show}\isamarkupfalse%
\ {\isachardoublequoteopen}{\isacharquery}{\kern0pt}left\ {\isasymsubseteq}\ {\isacharquery}{\kern0pt}right{\isachardoublequoteclose}\isanewline
\ \ \isacommand{next}\isamarkupfalse%
\isanewline
\ \ \isacommand{show}\isamarkupfalse%
\ {\isachardoublequoteopen}{\isacharquery}{\kern0pt}right\ {\isasymsubseteq}\ {\isacharquery}{\kern0pt}left{\isachardoublequoteclose}\isanewline
\ \ \isacommand{qed}\isamarkupfalse%
%
\endisatagproof
{\isafoldproof}%
%
\isadelimproof
\isanewline
%
\endisadelimproof
\isanewline
\isacommand{lemma}\isamarkupfalse%
\ {\isachardoublequoteopen}A\ {\isasyminter}\ {\isacharparenleft}{\kern0pt}B\ {\isasymunion}\ C{\isacharparenright}{\kern0pt}\ {\isacharequal}{\kern0pt}\ {\isacharparenleft}{\kern0pt}A\ {\isasyminter}\ B{\isacharparenright}{\kern0pt}\ {\isasymunion}\ {\isacharparenleft}{\kern0pt}A\ {\isasyminter}\ C{\isacharparenright}{\kern0pt}{\isachardoublequoteclose}\ {\isacharparenleft}{\kern0pt}\isakeyword{is}\ {\isachardoublequoteopen}{\isacharquery}{\kern0pt}left\ {\isacharequal}{\kern0pt}\ {\isacharquery}{\kern0pt}right{\isachardoublequoteclose}{\isacharparenright}{\kern0pt}\isanewline
\ \ %
\isadelimproof
%
\endisadelimproof
%
\isatagproof
%
\endisatagproof
{\isafoldproof}%
%
\isadelimproof
%
\endisadelimproof
\isanewline
\isacommand{lemma}\isamarkupfalse%
\ {\isachardoublequoteopen}A\ {\isacharminus}{\kern0pt}\ {\isacharparenleft}{\kern0pt}B\ {\isasyminter}\ C\ {\isacharparenright}{\kern0pt}\ {\isacharequal}{\kern0pt}\ {\isacharparenleft}{\kern0pt}A\ {\isacharminus}{\kern0pt}\ B{\isacharparenright}{\kern0pt}\ {\isasymunion}\ {\isacharparenleft}{\kern0pt}A\ {\isacharminus}{\kern0pt}\ C\ {\isacharparenright}{\kern0pt}{\isachardoublequoteclose}\ {\isacharparenleft}{\kern0pt}\isakeyword{is}\ {\isachardoublequoteopen}{\isacharquery}{\kern0pt}left\ {\isacharequal}{\kern0pt}\ {\isacharquery}{\kern0pt}right{\isachardoublequoteclose}{\isacharparenright}{\kern0pt}\isanewline
\ \ %
\isadelimproof
%
\endisadelimproof
%
\isatagproof
%
\endisatagproof
{\isafoldproof}%
%
\isadelimproof
%
\endisadelimproof
%
\end{exercise}
%
\isadelimtheory
%
\endisadelimtheory
%
\isatagtheory
%
\endisatagtheory
{\isafoldtheory}%
%
\isadelimtheory
%
\endisadelimtheory
%
\end{isabellebody}%
\endinput
%:%file=MyTheory.tex%:%
%:%18=8%:%
%:%21=10%:%
%:%23=12%:%
%:%24=12%:%
%:%25=13%:%
%:%26=13%:%
%:%27=14%:%
%:%28=14%:%
%:%29=15%:%
%:%30=15%:%
%:%31=16%:%
%:%32=16%:%
%:%33=17%:%
%:%34=17%:%
%:%35=19%:%
%:%37=22%:%
%:%40=24%:%
%:%44=26%:%
%:%45=27%:%
%:%47=29%:%
%:%48=29%:%
%:%49=30%:%
%:%64=32%:%
%:%65=33%:%
%:%67=35%:%
%:%68=35%:%
%:%75=36%:%
%:%76=36%:%
%:%77=37%:%
%:%78=37%:%
%:%79=38%:%
%:%80=38%:%
%:%81=39%:%
%:%82=39%:%
%:%83=40%:%
%:%84=40%:%
%:%85=41%:%
%:%86=41%:%
%:%87=42%:%
%:%87=43%:%
%:%88=43%:%
%:%89=44%:%
%:%90=44%:%
%:%91=45%:%
%:%92=45%:%
%:%93=46%:%
%:%93=47%:%
%:%103=49%:%
%:%104=50%:%
%:%106=52%:%
%:%107=52%:%
%:%114=53%:%
%:%115=53%:%
%:%116=54%:%
%:%117=54%:%
%:%118=55%:%
%:%118=56%:%
%:%119=56%:%
%:%120=57%:%
%:%121=57%:%
%:%122=58%:%
%:%122=59%:%
%:%128=59%:%
%:%131=60%:%
%:%132=61%:%
%:%133=61%:%
%:%134=62%:%
%:%147=63%:%
%:%148=64%:%
%:%149=64%:%
%:%150=65%:%
%:%164=67%:%
