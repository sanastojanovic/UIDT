%
\begin{isabellebody}%
\setisabellecontext{MyTheory}%
%
\isadelimtheory
%
\endisadelimtheory
%
\isatagtheory
%
\endisatagtheory
{\isafoldtheory}%
%
\isadelimtheory
%
\endisadelimtheory
%
\begin{exercise}[subtitle=Alternirajuća suma neparnih prirodnih brojeva]
%
\begin{isamarkuptext}%
Pokazati da važi:%
\end{isamarkuptext}\isamarkuptrue%
%
$$ - 1 + 3 - 5 + \ldots + (- 1)^n (2n - 1) = (- 1)^n n.$$
%
\begin{isamarkuptext}%
Primitivnom rekurzijom definisati funkciju \isa{alternirajuca{\isacharunderscore}{\kern0pt}suma\ {\isacharcolon}{\kern0pt}{\isacharcolon}{\kern0pt}\ {\isachardoublequote}{\kern0pt}nat\ {\isasymRightarrow}\ int{\isachardoublequote}{\kern0pt}} 
      koja računa alternirajucu sumu neparnih brojeva od \isa{{\isadigit{1}}} do \isa{{\isadigit{2}}n\ {\isacharminus}{\kern0pt}\ {\isadigit{1}}}, 
      tj. definisati funkciju koja računa levu stranu jednakosti.%
\end{isamarkuptext}\isamarkuptrue%
\isacommand{primrec}\isamarkupfalse%
\ alternirajuca{\isacharunderscore}{\kern0pt}suma\ {\isacharcolon}{\kern0pt}{\isacharcolon}{\kern0pt}\ {\isachardoublequoteopen}nat\ {\isasymRightarrow}\ int{\isachardoublequoteclose}\ \isakeyword{where}\isanewline
\ \ {\isachardoublequoteopen}alternirajuca{\isacharunderscore}{\kern0pt}suma\ {\isadigit{0}}\ {\isacharequal}{\kern0pt}\ undefined{\isachardoublequoteclose}\isanewline
{\isacharbar}{\kern0pt}\ {\isachardoublequoteopen}alternirajuca{\isacharunderscore}{\kern0pt}suma\ {\isacharparenleft}{\kern0pt}Suc\ n{\isacharparenright}{\kern0pt}\ {\isacharequal}{\kern0pt}\ undefined{\isachardoublequoteclose}%
\begin{isamarkuptext}%
Proveriti vrednost funkcije \isa{alternirajuca{\isacharunderscore}{\kern0pt}suma} za proizvoljan prirodni broj.%
\end{isamarkuptext}\isamarkuptrue%
%
\begin{isamarkuptext}%
Dokazati sledeću lemu induckijom koristeći metode za automatsko dokazivanje.%
\end{isamarkuptext}\isamarkuptrue%
\isacommand{lemma}\isamarkupfalse%
\ {\isachardoublequoteopen}alternirajuca{\isacharunderscore}{\kern0pt}suma\ n\ {\isacharequal}{\kern0pt}\ {\isacharparenleft}{\kern0pt}{\isacharminus}{\kern0pt}{\isadigit{1}}{\isacharparenright}{\kern0pt}{\isacharcircum}{\kern0pt}n\ {\isacharasterisk}{\kern0pt}\ int\ n{\isachardoublequoteclose}%
\isadelimproof
%
\endisadelimproof
%
\isatagproof
%
\endisatagproof
{\isafoldproof}%
%
\isadelimproof
%
\endisadelimproof
%
\begin{isamarkuptext}%
Dokazati sledeću lemu indukcijom raspisivanjem detaljnog Isar dokaza.%
\end{isamarkuptext}\isamarkuptrue%
\isacommand{lemma}\isamarkupfalse%
\ {\isachardoublequoteopen}alternirajuca{\isacharunderscore}{\kern0pt}suma\ n\ {\isacharequal}{\kern0pt}\ {\isacharparenleft}{\kern0pt}{\isacharminus}{\kern0pt}{\isadigit{1}}{\isacharparenright}{\kern0pt}{\isacharcircum}{\kern0pt}n\ {\isacharasterisk}{\kern0pt}\ int\ n{\isachardoublequoteclose}%
\isadelimproof
%
\endisadelimproof
%
\isatagproof
%
\endisatagproof
{\isafoldproof}%
%
\isadelimproof
%
\endisadelimproof
%
\end{exercise}
%
\begin{exercise}[subtitle=Množenje matrica]
%
\begin{isamarkuptext}%
Pokazati da važi sledeća jednakost:%
\end{isamarkuptext}\isamarkuptrue%
%
$$
\begin{pmatrix}
1 & 1\\
0 & 1
\end{pmatrix}^n
=
\begin{pmatrix}
1 & n\\
0 & 1
\end{pmatrix},
n \in \mathbb{N}.
$$
%
\begin{isamarkuptext}%
Definisati tip \isa{mat{\isadigit{2}}} koji predstavlja jednu \isa{{\isadigit{2}}{\isasymtimes}{\isadigit{2}}} matricu prirodnih brojeva.
      Tip \isa{mat{\isadigit{2}}} definisati kao skraćenicu uređene četvorke prirodnih brojeva.
      Uređena četvorka odgovara \isa{{\isadigit{2}}{\isasymtimes}{\isadigit{2}}} matrici kao%
\end{isamarkuptext}\isamarkuptrue%
%
$$
(a, b, c, d) \equiv
\begin{pmatrix}
a & b\\
c & d
\end{pmatrix}.
$$
%
\begin{isamarkuptext}%
Definisati konstantu \isa{eye\ {\isacharcolon}{\kern0pt}{\isacharcolon}{\kern0pt}\ mat{\isadigit{2}}}, koja predstavlja jediničnu matricu.%
\end{isamarkuptext}\isamarkuptrue%
%
\begin{isamarkuptext}%
Definisati funkciju \isa{mat{\isacharunderscore}{\kern0pt}mul\ {\isacharcolon}{\kern0pt}{\isacharcolon}{\kern0pt}\ mat{\isadigit{2}}\ {\isasymRightarrow}\ mat{\isadigit{2}}\ {\isasymRightarrow}\ mat{\isadigit{2}}}, koja množi dve matrice.%
\end{isamarkuptext}\isamarkuptrue%
\isacommand{fun}\isamarkupfalse%
\ mat{\isacharunderscore}{\kern0pt}mul\ \isakeyword{where}\isanewline
\ \ {\isachardoublequoteopen}mat{\isacharunderscore}{\kern0pt}mul\ {\isacharparenleft}{\kern0pt}a{\isadigit{1}}{\isacharcomma}{\kern0pt}\ b{\isadigit{1}}{\isacharcomma}{\kern0pt}\ c{\isadigit{1}}{\isacharcomma}{\kern0pt}\ d{\isadigit{1}}{\isacharparenright}{\kern0pt}\ {\isacharparenleft}{\kern0pt}a{\isadigit{2}}{\isacharcomma}{\kern0pt}\ b{\isadigit{2}}{\isacharcomma}{\kern0pt}\ c{\isadigit{2}}{\isacharcomma}{\kern0pt}\ d{\isadigit{2}}{\isacharparenright}{\kern0pt}\ {\isacharequal}{\kern0pt}\ undefined{\isachardoublequoteclose}%
\begin{isamarkuptext}%
Definisati funkciju \isa{mat{\isacharunderscore}{\kern0pt}pow\ {\isacharcolon}{\kern0pt}{\isacharcolon}{\kern0pt}\ mat{\isadigit{2}}\ {\isasymRightarrow}\ nat\ {\isasymRightarrow}\ mat{\isadigit{2}}}, koja stepenuje matricu.%
\end{isamarkuptext}\isamarkuptrue%
\isacommand{fun}\isamarkupfalse%
\ mat{\isacharunderscore}{\kern0pt}pow\ \isakeyword{where}\isanewline
\ \ {\isachardoublequoteopen}mat{\isacharunderscore}{\kern0pt}pow\ {\isacharunderscore}{\kern0pt}\ {\isacharunderscore}{\kern0pt}\ {\isacharequal}{\kern0pt}\ undefined{\isachardoublequoteclose}%
\begin{isamarkuptext}%
Dokazati sledeću lemu koristeći metode za automatsko dokazivanje.%
\end{isamarkuptext}\isamarkuptrue%
\isacommand{lemma}\isamarkupfalse%
\ {\isachardoublequoteopen}mat{\isacharunderscore}{\kern0pt}pow\ {\isacharparenleft}{\kern0pt}{\isadigit{1}}{\isacharcomma}{\kern0pt}\ {\isadigit{1}}{\isacharcomma}{\kern0pt}\ {\isadigit{0}}{\isacharcomma}{\kern0pt}\ {\isadigit{1}}{\isacharparenright}{\kern0pt}\ n\ {\isacharequal}{\kern0pt}\ {\isacharparenleft}{\kern0pt}{\isadigit{1}}{\isacharcomma}{\kern0pt}\ n{\isacharcomma}{\kern0pt}\ {\isadigit{0}}{\isacharcomma}{\kern0pt}\ {\isadigit{1}}{\isacharparenright}{\kern0pt}{\isachardoublequoteclose}%
\isadelimproof
%
\endisadelimproof
%
\isatagproof
%
\endisatagproof
{\isafoldproof}%
%
\isadelimproof
%
\endisadelimproof
%
\begin{isamarkuptext}%
Dokazati sledeću lemu indukcijom raspisivanjem detaljnog Isar dokaza.%
\end{isamarkuptext}\isamarkuptrue%
\isacommand{lemma}\isamarkupfalse%
\ {\isachardoublequoteopen}mat{\isacharunderscore}{\kern0pt}pow\ {\isacharparenleft}{\kern0pt}{\isadigit{1}}{\isacharcomma}{\kern0pt}\ {\isadigit{1}}{\isacharcomma}{\kern0pt}\ {\isadigit{0}}{\isacharcomma}{\kern0pt}\ {\isadigit{1}}{\isacharparenright}{\kern0pt}\ n\ {\isacharequal}{\kern0pt}\ {\isacharparenleft}{\kern0pt}{\isadigit{1}}{\isacharcomma}{\kern0pt}\ n{\isacharcomma}{\kern0pt}\ {\isadigit{0}}{\isacharcomma}{\kern0pt}\ {\isadigit{1}}{\isacharparenright}{\kern0pt}{\isachardoublequoteclose}%
\isadelimproof
%
\endisadelimproof
%
\isatagproof
%
\endisatagproof
{\isafoldproof}%
%
\isadelimproof
%
\endisadelimproof
%
\end{exercise}
%
\begin{exercise}[subtitle=Deljivost]
%
\begin{isamarkuptext}%
Pokazati sledeću lemu.\\
      \isa{Savet}: Obrisati \isa{One{\isacharunderscore}{\kern0pt}nat{\isacharunderscore}{\kern0pt}def} i \isa{algebra{\isacharunderscore}{\kern0pt}simps} iz \isa{simp}-a u 
      finalnom koraku dokaza.%
\end{isamarkuptext}\isamarkuptrue%
\isacommand{lemma}\isamarkupfalse%
\ \isanewline
\ \ \isakeyword{fixes}\ n{\isacharcolon}{\kern0pt}{\isacharcolon}{\kern0pt}nat\isanewline
\ \ \isakeyword{shows}\ {\isachardoublequoteopen}{\isacharparenleft}{\kern0pt}{\isadigit{6}}{\isacharcolon}{\kern0pt}{\isacharcolon}{\kern0pt}nat{\isacharparenright}{\kern0pt}\ dvd\ n\ {\isacharasterisk}{\kern0pt}\ {\isacharparenleft}{\kern0pt}n\ {\isacharplus}{\kern0pt}\ {\isadigit{1}}{\isacharparenright}{\kern0pt}\ {\isacharasterisk}{\kern0pt}\ {\isacharparenleft}{\kern0pt}{\isadigit{2}}\ {\isacharasterisk}{\kern0pt}\ n\ {\isacharplus}{\kern0pt}\ {\isadigit{1}}{\isacharparenright}{\kern0pt}{\isachardoublequoteclose}%
\isadelimproof
%
\endisadelimproof
%
\isatagproof
%
\endisatagproof
{\isafoldproof}%
%
\isadelimproof
%
\endisadelimproof
%
\end{exercise}
%
\begin{exercise}[subtitle=Nejednakost]
%
\begin{isamarkuptext}%
Pokazati da za svaki prirodan broj \isa{n\ {\isachargreater}{\kern0pt}\ {\isadigit{2}}} važi \isa{n\isactrlsup {\isadigit{2}}\ {\isachargreater}{\kern0pt}\ {\isadigit{2}}\ {\isacharasterisk}{\kern0pt}\ n\ {\isacharplus}{\kern0pt}\ {\isadigit{1}}}.\\
     \isa{Savet}: Iskoristiti \isa{nat{\isacharunderscore}{\kern0pt}induct{\isacharunderscore}{\kern0pt}at{\isacharunderscore}{\kern0pt}least} kao pravilo indukcije i 
              lemu \isa{power{\isadigit{2}}{\isacharunderscore}{\kern0pt}eq{\isacharunderscore}{\kern0pt}square}.%
\end{isamarkuptext}\isamarkuptrue%
\isacommand{thm}\isamarkupfalse%
\ nat{\isacharunderscore}{\kern0pt}induct{\isacharunderscore}{\kern0pt}at{\isacharunderscore}{\kern0pt}least\isanewline
\isacommand{thm}\isamarkupfalse%
\ power{\isadigit{2}}{\isacharunderscore}{\kern0pt}eq{\isacharunderscore}{\kern0pt}square\isanewline
\isanewline
\isacommand{lemma}\isamarkupfalse%
\ n{\isadigit{2}}{\isacharunderscore}{\kern0pt}{\isadigit{2}}n{\isacharcolon}{\kern0pt}\isanewline
\ \ \isakeyword{fixes}\ n{\isacharcolon}{\kern0pt}{\isacharcolon}{\kern0pt}nat\isanewline
\ \ \isakeyword{assumes}\ {\isachardoublequoteopen}n\ {\isasymge}\ {\isadigit{3}}{\isachardoublequoteclose}\isanewline
\ \ \isakeyword{shows}\ {\isachardoublequoteopen}n\isactrlsup {\isadigit{2}}\ {\isachargreater}{\kern0pt}\ {\isadigit{2}}\ {\isacharasterisk}{\kern0pt}\ n\ {\isacharplus}{\kern0pt}\ {\isadigit{1}}{\isachardoublequoteclose}\isanewline
%
\isadelimproof
\ \ %
\endisadelimproof
%
\isatagproof
\isacommand{using}\isamarkupfalse%
\ assms%
\endisatagproof
{\isafoldproof}%
%
\isadelimproof
%
\endisadelimproof
%
\end{exercise}
%
\isadelimtheory
%
\endisadelimtheory
%
\isatagtheory
%
\endisatagtheory
{\isafoldtheory}%
%
\isadelimtheory
%
\endisadelimtheory
%
\end{isabellebody}%
\endinput
%:%file=MyTheory.tex%:%
%:%18=8%:%
%:%21=10%:%
%:%24=12%:%
%:%27=14%:%
%:%28=15%:%
%:%29=16%:%
%:%31=18%:%
%:%32=18%:%
%:%33=19%:%
%:%34=20%:%
%:%36=22%:%
%:%40=24%:%
%:%42=26%:%
%:%43=26%:%
%:%58=29%:%
%:%60=31%:%
%:%61=31%:%
%:%75=34%:%
%:%77=36%:%
%:%80=38%:%
%:%83=41%:%
%:%84=42%:%
%:%85=43%:%
%:%86=44%:%
%:%87=45%:%
%:%88=46%:%
%:%89=47%:%
%:%90=48%:%
%:%91=49%:%
%:%92=50%:%
%:%93=51%:%
%:%94=52%:%
%:%97=54%:%
%:%98=55%:%
%:%99=56%:%
%:%102=59%:%
%:%103=60%:%
%:%104=61%:%
%:%105=62%:%
%:%106=63%:%
%:%107=64%:%
%:%108=65%:%
%:%111=67%:%
%:%115=69%:%
%:%117=71%:%
%:%118=71%:%
%:%119=72%:%
%:%121=74%:%
%:%123=76%:%
%:%124=76%:%
%:%125=77%:%
%:%127=79%:%
%:%129=81%:%
%:%130=81%:%
%:%145=84%:%
%:%147=86%:%
%:%148=86%:%
%:%162=89%:%
%:%164=91%:%
%:%167=93%:%
%:%168=94%:%
%:%169=95%:%
%:%171=97%:%
%:%172=97%:%
%:%173=98%:%
%:%174=99%:%
%:%188=102%:%
%:%190=104%:%
%:%193=106%:%
%:%194=107%:%
%:%195=108%:%
%:%197=110%:%
%:%198=110%:%
%:%199=111%:%
%:%200=111%:%
%:%201=112%:%
%:%202=113%:%
%:%203=113%:%
%:%204=114%:%
%:%205=115%:%
%:%206=116%:%
%:%209=117%:%
%:%213=117%:%
%:%214=117%:%
%:%222=120%:%
