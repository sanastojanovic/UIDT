%
\begin{isabellebody}%
\setisabellecontext{MyTheory}%
%
\isadelimtheory
%
\endisadelimtheory
%
\isatagtheory
%
\endisatagtheory
{\isafoldtheory}%
%
\isadelimtheory
%
\endisadelimtheory
%
\begin{exercise}[subtitle=Isar dokazi u logici prvog reda.]
\isacommand{lemma}\isamarkupfalse%
\ \isanewline
\ \ \isakeyword{assumes}\ {\isachardoublequoteopen}{\isacharparenleft}{\kern0pt}{\isasymexists}\ x{\isachardot}{\kern0pt}\ P\ x{\isacharparenright}{\kern0pt}{\isachardoublequoteclose}\isanewline
\ \ \ \ \ \ \isakeyword{and}\ {\isachardoublequoteopen}{\isacharparenleft}{\kern0pt}{\isasymforall}\ x{\isachardot}{\kern0pt}\ P\ x\ {\isasymlongrightarrow}\ Q\ x{\isacharparenright}{\kern0pt}{\isachardoublequoteclose}\isanewline
\ \ \ \ \isakeyword{shows}\ {\isachardoublequoteopen}{\isacharparenleft}{\kern0pt}{\isasymexists}\ x{\isachardot}{\kern0pt}\ Q\ x{\isacharparenright}{\kern0pt}{\isachardoublequoteclose}%
\isadelimproof
%
\endisadelimproof
%
\isatagproof
%
\endisatagproof
{\isafoldproof}%
%
\isadelimproof
%
\endisadelimproof
\isanewline
\isacommand{lemma}\isamarkupfalse%
\isanewline
\ \ \isakeyword{assumes}\ {\isachardoublequoteopen}{\isasymforall}\ c{\isachardot}{\kern0pt}\ Man\ c\ {\isasymlongrightarrow}\ Mortal\ c{\isachardoublequoteclose}\isanewline
\ \ \ \ \ \ \isakeyword{and}\ {\isachardoublequoteopen}{\isasymforall}\ g{\isachardot}{\kern0pt}\ Greek\ g\ {\isasymlongrightarrow}\ Man\ g{\isachardoublequoteclose}\isanewline
\ \ \ \ \isakeyword{shows}\ {\isachardoublequoteopen}{\isasymforall}\ a{\isachardot}{\kern0pt}\ Greek\ a\ {\isasymlongrightarrow}\ Mortal\ a{\isachardoublequoteclose}%
\isadelimproof
%
\endisadelimproof
%
\isatagproof
%
\endisatagproof
{\isafoldproof}%
%
\isadelimproof
%
\endisadelimproof
%
\begin{isamarkuptext}%
Dodatni primeri:%
\end{isamarkuptext}\isamarkuptrue%
\isacommand{lemma}\isamarkupfalse%
\isanewline
\ \ \isakeyword{assumes}\ {\isachardoublequoteopen}{\isasymforall}\ a{\isachardot}{\kern0pt}\ P\ a\ {\isasymlongrightarrow}\ Q\ a{\isachardoublequoteclose}\isanewline
\ \ \ \ \ \ \isakeyword{and}\ {\isachardoublequoteopen}{\isasymforall}\ b{\isachardot}{\kern0pt}\ P\ b{\isachardoublequoteclose}\isanewline
\ \ \ \ \isakeyword{shows}\ {\isachardoublequoteopen}{\isasymforall}\ x{\isachardot}{\kern0pt}\ Q\ x{\isachardoublequoteclose}%
\isadelimproof
%
\endisadelimproof
%
\isatagproof
%
\endisatagproof
{\isafoldproof}%
%
\isadelimproof
%
\endisadelimproof
\isanewline
\isacommand{lemma}\isamarkupfalse%
\isanewline
\ \ \isakeyword{assumes}\ {\isachardoublequoteopen}{\isasymexists}\ x{\isachardot}{\kern0pt}\ A\ x\ {\isasymor}\ B\ x{\isachardoublequoteclose}\isanewline
\ \ \ \ \isakeyword{shows}\ {\isachardoublequoteopen}{\isacharparenleft}{\kern0pt}{\isasymexists}\ x{\isachardot}{\kern0pt}\ A\ x{\isacharparenright}{\kern0pt}\ {\isasymor}\ {\isacharparenleft}{\kern0pt}{\isasymexists}\ x{\isachardot}{\kern0pt}\ B\ x{\isacharparenright}{\kern0pt}{\isachardoublequoteclose}%
\isadelimproof
%
\endisadelimproof
%
\isatagproof
%
\endisatagproof
{\isafoldproof}%
%
\isadelimproof
%
\endisadelimproof
\isanewline
\isacommand{lemma}\isamarkupfalse%
\isanewline
\ \ \isakeyword{assumes}\ {\isachardoublequoteopen}{\isasymforall}\ x{\isachardot}{\kern0pt}\ A\ x\ {\isasymlongrightarrow}\ {\isasymnot}\ B\ x{\isachardoublequoteclose}\isanewline
\ \ \ \ \isakeyword{shows}\ {\isachardoublequoteopen}{\isasymnot}\ {\isacharparenleft}{\kern0pt}{\isasymexists}\ x{\isachardot}{\kern0pt}\ A\ x\ {\isasymand}\ B\ x{\isacharparenright}{\kern0pt}{\isachardoublequoteclose}%
\isadelimproof
%
\endisadelimproof
%
\isatagproof
%
\endisatagproof
{\isafoldproof}%
%
\isadelimproof
%
\endisadelimproof
%
\begin{isamarkuptext}%
Formulisati i dokazati naredna tvrđenja u Isar jaziku:%
\end{isamarkuptext}\isamarkuptrue%
%
\begin{isamarkuptext}%
Ako za svaki broj koji nije paran važi da je neparan;\\
      i ako za svaki neparan broj važi da nije paran;\\
      pokazati da onda za svaki broj važi da je ili paran ili neparan.%
\end{isamarkuptext}\isamarkuptrue%
%
\begin{isamarkuptext}%
Ako svaki konj ima potkovice;\\
      i ako ne postoji čovek koji ima potkovice;\\
      i ako znamo da postoji makar jedan čovek;\\
      dokazati da postoji čovek koji nije konj.%
\end{isamarkuptext}\isamarkuptrue%
%
\end{exercise}
%
\begin{exercise}[subtitle=Pravilo ccontr i classical.]
%
\begin{isamarkuptext}%
Dokazati u Isar jeziku naredna tvrđenja pomoću pravila \isa{ccontr}.%
\end{isamarkuptext}\isamarkuptrue%
\isacommand{lemma}\isamarkupfalse%
\ {\isachardoublequoteopen}{\isasymnot}\ {\isacharparenleft}{\kern0pt}A\ {\isasymand}\ B{\isacharparenright}{\kern0pt}\ {\isasymlongrightarrow}\ {\isasymnot}\ A\ {\isasymor}\ {\isasymnot}\ B{\isachardoublequoteclose}%
\isadelimproof
%
\endisadelimproof
%
\isatagproof
%
\endisatagproof
{\isafoldproof}%
%
\isadelimproof
%
\endisadelimproof
%
\begin{isamarkuptext}%
Dodatni primer:%
\end{isamarkuptext}\isamarkuptrue%
\isacommand{lemma}\isamarkupfalse%
\ {\isachardoublequoteopen}{\isacharparenleft}{\kern0pt}{\isacharparenleft}{\kern0pt}P\ {\isasymlongrightarrow}\ Q{\isacharparenright}{\kern0pt}\ {\isasymlongrightarrow}\ P{\isacharparenright}{\kern0pt}\ {\isasymlongrightarrow}\ P{\isachardoublequoteclose}%
\isadelimproof
%
\endisadelimproof
%
\isatagproof
%
\endisatagproof
{\isafoldproof}%
%
\isadelimproof
%
\endisadelimproof
%
\begin{isamarkuptext}%
Dokazati u Isar jeziku naredna tvrđenja pomoću pravila \isa{classical}.%
\end{isamarkuptext}\isamarkuptrue%
\isacommand{lemma}\isamarkupfalse%
\ {\isachardoublequoteopen}P\ {\isasymor}\ {\isasymnot}\ P{\isachardoublequoteclose}%
\isadelimproof
%
\endisadelimproof
%
\isatagproof
%
\endisatagproof
{\isafoldproof}%
%
\isadelimproof
%
\endisadelimproof
%
\begin{isamarkuptext}%
Dodatni primer:%
\end{isamarkuptext}\isamarkuptrue%
\isacommand{lemma}\isamarkupfalse%
\isanewline
\ \ \isakeyword{assumes}\ {\isachardoublequoteopen}{\isasymnot}\ {\isacharparenleft}{\kern0pt}{\isasymforall}\ x{\isachardot}{\kern0pt}\ P\ x{\isacharparenright}{\kern0pt}{\isachardoublequoteclose}\isanewline
\ \ \ \ \isakeyword{shows}\ {\isachardoublequoteopen}{\isasymexists}\ x{\isachardot}{\kern0pt}\ {\isasymnot}\ P\ x{\isachardoublequoteclose}%
\isadelimproof
%
\endisadelimproof
%
\isatagproof
%
\endisatagproof
{\isafoldproof}%
%
\isadelimproof
%
\endisadelimproof
%
\end{exercise}
%
\begin{exercise}[subtitle=Logčki lavirinti.]
%
\begin{isamarkuptext}%
Svaka osoba daje potvrdan odgovor na pitanje: \isa{Da\ li\ si\ ti\ vitez{\isacharquery}{\kern0pt}}%
\end{isamarkuptext}\isamarkuptrue%
\isacommand{lemma}\isamarkupfalse%
\ no{\isacharunderscore}{\kern0pt}one{\isacharunderscore}{\kern0pt}admits{\isacharunderscore}{\kern0pt}knave{\isacharcolon}{\kern0pt}\isanewline
\ \ \isakeyword{assumes}\ {\isachardoublequoteopen}k\ {\isasymlongleftrightarrow}\ {\isacharparenleft}{\kern0pt}k\ {\isasymlongleftrightarrow}\ ans{\isacharparenright}{\kern0pt}{\isachardoublequoteclose}\isanewline
\ \ \ \ \isakeyword{shows}\ ans%
\isadelimproof
%
\endisadelimproof
%
\isatagproof
%
\endisatagproof
{\isafoldproof}%
%
\isadelimproof
%
\endisadelimproof
%
\begin{isamarkuptext}%
Abercrombie je sreo tri stanovnika, koje ćemo zvati A, B i C. 
      Pitao je A: Jesi li ti vitez ili podanik?
      On je odgovorio, ali tako nejasno da Abercrombie nije mogao shvati 
      što je rekao. 
      Zatim je upitao B: Šta je rekao? 
      B odgovori: Rekao je da je podanik.
      U tom trenutku, C se ubacio i rekao: Ne verujte u to; to je laž! 
      Je li C bio vitez ili podanik?%
\end{isamarkuptext}\isamarkuptrue%
\isacommand{lemma}\isamarkupfalse%
\ Smullyan{\isacharunderscore}{\kern0pt}{\isadigit{1}}{\isacharunderscore}{\kern0pt}{\isadigit{1}}{\isacharcolon}{\kern0pt}\isanewline
\ \ \isakeyword{assumes}\ {\isachardoublequoteopen}kA\ {\isasymlongleftrightarrow}\ {\isacharparenleft}{\kern0pt}kA\ {\isasymlongleftrightarrow}\ ansA{\isacharparenright}{\kern0pt}{\isachardoublequoteclose}\isanewline
\ \ \ \ \ \ \isakeyword{and}\ {\isachardoublequoteopen}kB\ {\isasymlongleftrightarrow}\ {\isasymnot}\ ansA{\isachardoublequoteclose}\isanewline
\ \ \ \ \ \ \isakeyword{and}\ {\isachardoublequoteopen}kC\ {\isasymlongleftrightarrow}\ {\isasymnot}\ kB{\isachardoublequoteclose}\isanewline
\ \ \ \ \isakeyword{shows}\ kC%
\isadelimproof
%
\endisadelimproof
%
\isatagproof
%
\endisatagproof
{\isafoldproof}%
%
\isadelimproof
%
\endisadelimproof
%
\begin{isamarkuptext}%
Abercrombie nije pitao A da li je on vitez ili podanik 
      (jer bi unapred znao koji će odgovor dobiti), 
      već je pitao A koliko od njih trojice su bili vitezovi. 
      Opet je A odgovorio nejasno, pa je Abercrombie upitao B što je A rekao. 
      B je tada rekao da je A rekao da su tačno njih dvojica podanici. 
      Tada je, kao i prije, C tvrdio da B laže. 
      Je li je sada moguće utvrditi da li je C vitez ili podanik?%
\end{isamarkuptext}\isamarkuptrue%
\isacommand{definition}\isamarkupfalse%
\ exactly{\isacharunderscore}{\kern0pt}two\ {\isacharcolon}{\kern0pt}{\isacharcolon}{\kern0pt}\ {\isachardoublequoteopen}bool\ {\isasymRightarrow}\ bool\ {\isasymRightarrow}\ bool\ {\isasymRightarrow}\ bool{\isachardoublequoteclose}\ \isakeyword{where}\isanewline
\ \ {\isachardoublequoteopen}exactly{\isacharunderscore}{\kern0pt}two\ A\ B\ C\ {\isasymlongleftrightarrow}\ {\isacharparenleft}{\kern0pt}{\isacharparenleft}{\kern0pt}A\ {\isasymand}\ B{\isacharparenright}{\kern0pt}\ {\isasymor}\ {\isacharparenleft}{\kern0pt}A\ {\isasymand}\ C{\isacharparenright}{\kern0pt}\ {\isasymor}\ {\isacharparenleft}{\kern0pt}B\ {\isasymand}\ C{\isacharparenright}{\kern0pt}{\isacharparenright}{\kern0pt}\ {\isasymand}\ {\isasymnot}\ {\isacharparenleft}{\kern0pt}A\ {\isasymand}\ B\ {\isasymand}\ C{\isacharparenright}{\kern0pt}{\isachardoublequoteclose}\isanewline
\isanewline
\isacommand{lemma}\isamarkupfalse%
\ Smullyan{\isacharunderscore}{\kern0pt}{\isadigit{1}}{\isacharunderscore}{\kern0pt}{\isadigit{2}}{\isacharcolon}{\kern0pt}\isanewline
\ \ \isakeyword{assumes}\ {\isachardoublequoteopen}kB\ {\isasymlongleftrightarrow}\ {\isacharparenleft}{\kern0pt}kA\ {\isasymlongleftrightarrow}\ exactly{\isacharunderscore}{\kern0pt}two\ {\isacharparenleft}{\kern0pt}{\isasymnot}\ kA{\isacharparenright}{\kern0pt}\ {\isacharparenleft}{\kern0pt}{\isasymnot}\ kB{\isacharparenright}{\kern0pt}\ {\isacharparenleft}{\kern0pt}{\isasymnot}\ kC{\isacharparenright}{\kern0pt}{\isacharparenright}{\kern0pt}{\isachardoublequoteclose}\isanewline
\ \ \ \ \ \ \isakeyword{and}\ {\isachardoublequoteopen}kC\ {\isasymlongleftrightarrow}\ {\isasymnot}\ kB{\isachardoublequoteclose}\isanewline
\ \ \ \ \isakeyword{shows}\ {\isachardoublequoteopen}kC{\isachardoublequoteclose}%
\isadelimproof
%
\endisadelimproof
%
\isatagproof
%
\endisatagproof
{\isafoldproof}%
%
\isadelimproof
%
\endisadelimproof
%
\begin{isamarkuptext}%
Abercrombie je sreo samo dva stanovnika A i B. 
      A je izjavio: Obojica smo podanici. 
      Da li možemo da zaključimo šta je A a šta je B?%
\end{isamarkuptext}\isamarkuptrue%
\isacommand{lemma}\isamarkupfalse%
\ Smullyan{\isacharunderscore}{\kern0pt}{\isadigit{1}}{\isacharunderscore}{\kern0pt}{\isadigit{3}}{\isacharcolon}{\kern0pt}\isanewline
\ \ {\isachardoublequoteopen}x{\isachardoublequoteclose}%
\isadelimproof
%
\endisadelimproof
%
\isatagproof
%
\endisatagproof
{\isafoldproof}%
%
\isadelimproof
%
\endisadelimproof
%
\begin{isamarkuptext}%
A nije rekao: Obojica smo podanici. 
      Ono što je on rekao je: Bar jedan od nas je podanik. 
      Ako je ova verzija odgovora tačna, šta su A i B?%
\end{isamarkuptext}\isamarkuptrue%
\isacommand{lemma}\isamarkupfalse%
\ Smullyan{\isacharunderscore}{\kern0pt}{\isadigit{1}}{\isacharunderscore}{\kern0pt}{\isadigit{4}}{\isacharcolon}{\kern0pt}\isanewline
\ \ {\isachardoublequoteopen}x{\isachardoublequoteclose}%
\isadelimproof
%
\endisadelimproof
%
\isatagproof
%
\endisatagproof
{\isafoldproof}%
%
\isadelimproof
%
\endisadelimproof
%
\begin{isamarkuptext}%
A je rekao: Svi smo istog tipa tj. 
      ili smo svi vitezovi ili podanici. 
      Ako je ova verzija priče tačna, 
      šta možemo zaključiti o A i B?%
\end{isamarkuptext}\isamarkuptrue%
\isacommand{lemma}\isamarkupfalse%
\ Smullyan{\isacharunderscore}{\kern0pt}{\isadigit{1}}{\isacharunderscore}{\kern0pt}{\isadigit{5}}{\isacharcolon}{\kern0pt}\ \isanewline
\ \ {\isachardoublequoteopen}x{\isachardoublequoteclose}%
\isadelimproof
%
\endisadelimproof
%
\isatagproof
%
\endisatagproof
{\isafoldproof}%
%
\isadelimproof
%
\endisadelimproof
%
\begin{isamarkuptext}%
Primetiti da ova lema odgovara lemi \isa{no{\isacharunderscore}{\kern0pt}one{\isacharunderscore}{\kern0pt}admits{\isacharunderscore}{\kern0pt}knave}. 
      Zašto se ne može ništa zaključiti o osobi A?%
\end{isamarkuptext}\isamarkuptrue%
%
\end{exercise}
%
\isadelimtheory
%
\endisadelimtheory
%
\isatagtheory
%
\endisatagtheory
{\isafoldtheory}%
%
\isadelimtheory
%
\endisadelimtheory
%
\end{isabellebody}%
\endinput
%:%file=MyTheory.tex%:%
%:%18=8%:%
%:%19=10%:%
%:%20=10%:%
%:%21=11%:%
%:%22=12%:%
%:%23=13%:%
%:%36=15%:%
%:%37=16%:%
%:%38=16%:%
%:%39=17%:%
%:%40=18%:%
%:%41=19%:%
%:%56=22%:%
%:%58=24%:%
%:%59=24%:%
%:%60=25%:%
%:%61=26%:%
%:%62=27%:%
%:%75=29%:%
%:%76=30%:%
%:%77=30%:%
%:%78=31%:%
%:%79=32%:%
%:%92=34%:%
%:%93=35%:%
%:%94=35%:%
%:%95=36%:%
%:%96=37%:%
%:%111=40%:%
%:%115=42%:%
%:%116=43%:%
%:%117=44%:%
%:%121=46%:%
%:%122=47%:%
%:%123=48%:%
%:%124=49%:%
%:%127=51%:%
%:%129=53%:%
%:%132=55%:%
%:%134=57%:%
%:%135=57%:%
%:%150=60%:%
%:%152=62%:%
%:%153=62%:%
%:%168=65%:%
%:%170=67%:%
%:%171=67%:%
%:%186=70%:%
%:%188=72%:%
%:%189=72%:%
%:%190=73%:%
%:%191=74%:%
%:%205=78%:%
%:%207=80%:%
%:%210=82%:%
%:%212=84%:%
%:%213=84%:%
%:%214=85%:%
%:%215=86%:%
%:%230=89%:%
%:%231=90%:%
%:%232=91%:%
%:%233=92%:%
%:%234=93%:%
%:%235=94%:%
%:%236=95%:%
%:%237=96%:%
%:%239=98%:%
%:%240=98%:%
%:%241=99%:%
%:%242=100%:%
%:%243=101%:%
%:%244=102%:%
%:%259=105%:%
%:%260=106%:%
%:%261=107%:%
%:%262=108%:%
%:%263=109%:%
%:%264=110%:%
%:%265=111%:%
%:%267=113%:%
%:%268=113%:%
%:%269=114%:%
%:%270=115%:%
%:%271=116%:%
%:%272=116%:%
%:%273=117%:%
%:%274=118%:%
%:%275=119%:%
%:%290=122%:%
%:%291=123%:%
%:%292=124%:%
%:%294=126%:%
%:%295=126%:%
%:%296=127%:%
%:%311=130%:%
%:%312=131%:%
%:%313=132%:%
%:%315=134%:%
%:%316=134%:%
%:%317=135%:%
%:%332=138%:%
%:%333=139%:%
%:%334=140%:%
%:%335=141%:%
%:%337=143%:%
%:%338=143%:%
%:%339=144%:%
%:%354=147%:%
%:%355=148%:%
%:%358=150%:%
