%
\begin{isabellebody}%
\setisabellecontext{MyTheory}%
%
\isadelimtheory
%
\endisadelimtheory
%
\isatagtheory
%
\endisatagtheory
{\isafoldtheory}%
%
\isadelimtheory
%
\endisadelimtheory
%
\begin{exercise}[subtitle=Zapisivanje logičkih formula (nastavak)]
%
\begin{isamarkuptext}%
(a) Zapisati sledeće rečenice u logici prvog reda i dokazati njihovu ispravnost.%
\end{isamarkuptext}\isamarkuptrue%
%
\begin{isamarkuptext}%
(a.1) Ako ”šta leti to ima krila i lagano je” 
            i ”šta pliva, to nema krila”, 
            onda ”šta pliva, to ne leti”%
\end{isamarkuptext}\isamarkuptrue%
%
\begin{isamarkuptext}%
(a.2) Ako postoji cipela koja u svakom trenutku odgovara svakoj nozi, 
            onda za svaku nogu postoji cipela koja joj u nekom trenutku odgovara 
            i za svaku nogu postoji trenutak takav da postoji cipela koja joj u tom 
            trenutku odgovara.%
\end{isamarkuptext}\isamarkuptrue%
%
\begin{isamarkuptext}%
(b) Pokazati da je rečenica P logička posledica rečenica P1, P2, P3.%
\end{isamarkuptext}\isamarkuptrue%
%
\begin{isamarkuptext}%
P:  Andrija voli da pleše.%
\end{isamarkuptext}\isamarkuptrue%
%
\begin{isamarkuptext}%
P1: Svako ko je srećan voli da peva.%
\end{isamarkuptext}\isamarkuptrue%
%
\begin{isamarkuptext}%
P2: Svako ko voli da peva, voli da pleše.%
\end{isamarkuptext}\isamarkuptrue%
%
\begin{isamarkuptext}%
P3: Andrija je srećan.%
\end{isamarkuptext}\isamarkuptrue%
%
\begin{isamarkuptext}%
P':  Svako dete voli da se igra.%
\end{isamarkuptext}\isamarkuptrue%
%
\begin{isamarkuptext}%
P1': Svaki dečak voli da se igra.%
\end{isamarkuptext}\isamarkuptrue%
%
\begin{isamarkuptext}%
P2': Svaka devojčica voli da se igra.%
\end{isamarkuptext}\isamarkuptrue%
%
\begin{isamarkuptext}%
P3': Dete je dečak ili je devojčica.%
\end{isamarkuptext}\isamarkuptrue%
%
\begin{isamarkuptext}%
(c) Na jeziku logike prvog reda zapisati sledeće rečenice i dokazati da su skupa nezadovoljive.%
\end{isamarkuptext}\isamarkuptrue%
%
\begin{isamarkuptext}%
- Svaka dva brata imaju zajedničkog roditelja.%
\end{isamarkuptext}\isamarkuptrue%
%
\begin{isamarkuptext}%
- Roditelj je stariji od deteta.%
\end{isamarkuptext}\isamarkuptrue%
%
\begin{isamarkuptext}%
- Postoje braća.%
\end{isamarkuptext}\isamarkuptrue%
%
\begin{isamarkuptext}%
- Nijedna osoba nije starija od druge.%
\end{isamarkuptext}\isamarkuptrue%
%
\end{exercise}
%
\begin{exercise}[subtitle=Silogizmi]
%
\begin{isamarkuptext}%
Barbara (AAA-1)%
\end{isamarkuptext}\isamarkuptrue%
%
\begin{isamarkuptext}%
All men are mortal. (MaP)%
\end{isamarkuptext}\isamarkuptrue%
%
\begin{isamarkuptext}%
All Greeks are men. (SaM)%
\end{isamarkuptext}\isamarkuptrue%
%
\begin{isamarkuptext}%
— All Greeks are mortal. (SaP)%
\end{isamarkuptext}\isamarkuptrue%
\isacommand{lemma}\isamarkupfalse%
\ Barbara{\isacharcolon}{\kern0pt}\ %
\isadelimproof
%
\endisadelimproof
%
\isatagproof
%
\endisatagproof
{\isafoldproof}%
%
\isadelimproof
%
\endisadelimproof
%
\begin{isamarkuptext}%
Celarent (EAE-1)%
\end{isamarkuptext}\isamarkuptrue%
%
\begin{isamarkuptext}%
Similar: Cesare (EAE-2)%
\end{isamarkuptext}\isamarkuptrue%
%
\begin{isamarkuptext}%
No reptiles have fur. (MeP)%
\end{isamarkuptext}\isamarkuptrue%
%
\begin{isamarkuptext}%
All snakes are reptiles. (SaM)%
\end{isamarkuptext}\isamarkuptrue%
%
\begin{isamarkuptext}%
— No snakes have fur. (SeP)%
\end{isamarkuptext}\isamarkuptrue%
\isacommand{lemma}\isamarkupfalse%
\ Celarent{\isacharcolon}{\kern0pt}\ %
\isadelimproof
%
\endisadelimproof
%
\isatagproof
%
\endisatagproof
{\isafoldproof}%
%
\isadelimproof
%
\endisadelimproof
%
\begin{isamarkuptext}%
Ferioque (EIO-1)%
\end{isamarkuptext}\isamarkuptrue%
%
\begin{isamarkuptext}%
No homework is fun. (MeP)%
\end{isamarkuptext}\isamarkuptrue%
%
\begin{isamarkuptext}%
Some reading is homework. (SiM)%
\end{isamarkuptext}\isamarkuptrue%
%
\begin{isamarkuptext}%
— Some reading is not fun. (SoP)%
\end{isamarkuptext}\isamarkuptrue%
\isacommand{lemma}\isamarkupfalse%
\ Ferioque{\isacharcolon}{\kern0pt}\ %
\isadelimproof
%
\endisadelimproof
%
\isatagproof
%
\endisatagproof
{\isafoldproof}%
%
\isadelimproof
%
\endisadelimproof
%
\begin{isamarkuptext}%
Bocardo (OAO-3)%
\end{isamarkuptext}\isamarkuptrue%
%
\begin{isamarkuptext}%
Some cats are not pets. (MoP)%
\end{isamarkuptext}\isamarkuptrue%
%
\begin{isamarkuptext}%
All cats are mammals. (MaS)%
\end{isamarkuptext}\isamarkuptrue%
%
\begin{isamarkuptext}%
— Some mammals are not pets. (SoP)%
\end{isamarkuptext}\isamarkuptrue%
\isacommand{lemma}\isamarkupfalse%
\ Bocardo{\isacharcolon}{\kern0pt}%
\isadelimproof
%
\endisadelimproof
%
\isatagproof
%
\endisatagproof
{\isafoldproof}%
%
\isadelimproof
%
\endisadelimproof
%
\begin{isamarkuptext}%
Barbari (AAI-1)%
\end{isamarkuptext}\isamarkuptrue%
%
\begin{isamarkuptext}%
All men are mortal. (MaP)%
\end{isamarkuptext}\isamarkuptrue%
%
\begin{isamarkuptext}%
All Greeks are men. (SaM)%
\end{isamarkuptext}\isamarkuptrue%
%
\begin{isamarkuptext}%
— Some Greeks are mortal. (SiP)%
\end{isamarkuptext}\isamarkuptrue%
\isacommand{lemma}\isamarkupfalse%
\ Barbari{\isacharcolon}{\kern0pt}\ %
\isadelimproof
%
\endisadelimproof
%
\isatagproof
%
\endisatagproof
{\isafoldproof}%
%
\isadelimproof
%
\endisadelimproof
%
\begin{isamarkuptext}%
Celaront (EAO-1)%
\end{isamarkuptext}\isamarkuptrue%
%
\begin{isamarkuptext}%
No reptiles have fur. (MeP)%
\end{isamarkuptext}\isamarkuptrue%
%
\begin{isamarkuptext}%
All snakes are reptiles. (SaM)%
\end{isamarkuptext}\isamarkuptrue%
%
\begin{isamarkuptext}%
— Some snakes have no fur. (SoP)%
\end{isamarkuptext}\isamarkuptrue%
\isacommand{lemma}\isamarkupfalse%
\ Celaront{\isacharcolon}{\kern0pt}%
\isadelimproof
%
\endisadelimproof
%
\isatagproof
%
\endisatagproof
{\isafoldproof}%
%
\isadelimproof
%
\endisadelimproof
%
\begin{isamarkuptext}%
Camestros (AEO-2)%
\end{isamarkuptext}\isamarkuptrue%
%
\begin{isamarkuptext}%
All horses have hooves. (PaM)%
\end{isamarkuptext}\isamarkuptrue%
%
\begin{isamarkuptext}%
No humans have hooves. (SeM)%
\end{isamarkuptext}\isamarkuptrue%
%
\begin{isamarkuptext}%
— Some humans are not horses. (SoP)%
\end{isamarkuptext}\isamarkuptrue%
\isacommand{lemma}\isamarkupfalse%
\ Camestros{\isacharcolon}{\kern0pt}%
\isadelimproof
%
\endisadelimproof
%
\isatagproof
%
\endisatagproof
{\isafoldproof}%
%
\isadelimproof
%
\endisadelimproof
%
\begin{isamarkuptext}%
Felapton (EAO-3)%
\end{isamarkuptext}\isamarkuptrue%
%
\begin{isamarkuptext}%
No flowers are animals. (MeP)%
\end{isamarkuptext}\isamarkuptrue%
%
\begin{isamarkuptext}%
All flowers are plants. (MaS)%
\end{isamarkuptext}\isamarkuptrue%
%
\begin{isamarkuptext}%
— Some plants are not animals. (SoP)%
\end{isamarkuptext}\isamarkuptrue%
\isacommand{lemma}\isamarkupfalse%
\ Felapton{\isacharcolon}{\kern0pt}\ %
\isadelimproof
%
\endisadelimproof
%
\isatagproof
%
\endisatagproof
{\isafoldproof}%
%
\isadelimproof
%
\endisadelimproof
%
\end{exercise}
%
\begin{exercise}[subtitle=Raymond M. Smullyan: Logical Labyrinths]
%
\begin{isamarkuptext}%
Edgar Aberkrombi je bio antropolog koji se interesovao za logiku i socijologiju
      laganja i govorenja istine. Jednog dana je odlučio da poseti ostrvo vitezova i podanika.
      Stanovnike ovog ostrva delimo na one koji uvek govore istinu \isa{vitezove} i
      one koji uvek govore laži \isa{podanike}. Dodatno, na ostrvu žive samo vitezovi i 
      podanici. Aberkrombi susreće stanovnike i želi da prepozna ko je od njih vitez, 
      a ko je podatnik.%
\end{isamarkuptext}\isamarkuptrue%
%
\begin{isamarkuptext}%
1. Svaka osoba će odgovoriti potvrdno na pitanje: Da li si ti vitez?%
\end{isamarkuptext}\isamarkuptrue%
\isacommand{lemma}\isamarkupfalse%
\ no{\isacharunderscore}{\kern0pt}one{\isacharunderscore}{\kern0pt}admit{\isacharunderscore}{\kern0pt}knaves{\isacharcolon}{\kern0pt}\ %
\isadelimproof
%
\endisadelimproof
%
\isatagproof
%
\endisatagproof
{\isafoldproof}%
%
\isadelimproof
%
\endisadelimproof
%
\begin{isamarkuptext}%
1.1 Aberkombi je razgovarao sa tri stanovnika ostrva, označimo ih sa A, B i C. 
          Pitao je stanovnika A: ”Da li si ti vitez ili podanik?” 
          A je odgovorio ali nerazgovetno 
          pa je Aberkombi pitao stanovnika B: ”Šta je A rekao?” 
          B je odgovorio: ”Rekao je da je on podanik.” 
          Tada se uključila i osoba C i rekla: ”Ne veruj mu, on laže!” 
          Da li je osoba C vitez ili podanik?%
\end{isamarkuptext}\isamarkuptrue%
\isacommand{lemma}\isamarkupfalse%
\ Smullyan{\isacharunderscore}{\kern0pt}{\isadigit{1}}{\isacharunderscore}{\kern0pt}{\isadigit{1}}{\isacharcolon}{\kern0pt}%
\isadelimproof
%
\endisadelimproof
%
\isatagproof
%
\endisatagproof
{\isafoldproof}%
%
\isadelimproof
%
\endisadelimproof
%
\begin{isamarkuptext}%
1.2 Aberkombi je pitao 
          stanovnika A koliko među njima trojicom ima podanika. 
          A je opet odgovorio nerazgovetno,
          tako da je Aberkombi pitao stanovnika B šta je A rekao. 
          B je rekao da je A rekao da su tačno dvojica podanici. 
          Ponovo je stanovnik C tvrdio da B laže. 
          Da li je u ovoj situaciji moguće odrediti da li je C vitez ili podanik?%
\end{isamarkuptext}\isamarkuptrue%
\isacommand{lemma}\isamarkupfalse%
\ Smullyan{\isacharunderscore}{\kern0pt}{\isadigit{1}}{\isacharunderscore}{\kern0pt}{\isadigit{2}}{\isacharcolon}{\kern0pt}%
\isadelimproof
%
\endisadelimproof
%
\isatagproof
%
\endisatagproof
{\isafoldproof}%
%
\isadelimproof
%
\endisadelimproof
%
\begin{isamarkuptext}%
1.3 Da li se zaključak prethodnog tvrđenja menja ako B promeni svoj odgovor 
          i kaže da je A rekao da su tačno dva od njih vitezovi?%
\end{isamarkuptext}\isamarkuptrue%
\isacommand{lemma}\isamarkupfalse%
\ Smullyan{\isacharunderscore}{\kern0pt}{\isadigit{1}}{\isacharunderscore}{\kern0pt}{\isadigit{3}}{\isacharcolon}{\kern0pt}%
\isadelimproof
%
\endisadelimproof
%
\isatagproof
%
\endisatagproof
{\isafoldproof}%
%
\isadelimproof
%
\endisadelimproof
%
\end{exercise}
%
\isadelimtheory
%
\endisadelimtheory
%
\isatagtheory
%
\endisatagtheory
{\isafoldtheory}%
%
\isadelimtheory
%
\endisadelimtheory
%
\end{isabellebody}%
\endinput
%:%file=MyTheory.tex%:%
%:%18=8%:%
%:%21=10%:%
%:%25=12%:%
%:%26=13%:%
%:%27=14%:%
%:%31=16%:%
%:%32=17%:%
%:%33=18%:%
%:%34=19%:%
%:%38=21%:%
%:%42=23%:%
%:%46=24%:%
%:%50=25%:%
%:%54=26%:%
%:%58=28%:%
%:%62=29%:%
%:%66=30%:%
%:%70=31%:%
%:%74=33%:%
%:%78=35%:%
%:%82=36%:%
%:%86=37%:%
%:%90=38%:%
%:%93=40%:%
%:%95=42%:%
%:%98=44%:%
%:%102=45%:%
%:%106=46%:%
%:%110=47%:%
%:%112=49%:%
%:%113=49%:%
%:%128=52%:%
%:%132=53%:%
%:%136=54%:%
%:%140=55%:%
%:%144=56%:%
%:%146=58%:%
%:%147=58%:%
%:%162=61%:%
%:%166=62%:%
%:%170=63%:%
%:%174=64%:%
%:%176=66%:%
%:%177=66%:%
%:%192=70%:%
%:%196=71%:%
%:%200=72%:%
%:%204=73%:%
%:%206=75%:%
%:%207=75%:%
%:%222=78%:%
%:%226=79%:%
%:%230=80%:%
%:%234=81%:%
%:%236=83%:%
%:%237=83%:%
%:%252=86%:%
%:%256=87%:%
%:%260=88%:%
%:%264=89%:%
%:%266=91%:%
%:%267=91%:%
%:%282=94%:%
%:%286=95%:%
%:%290=96%:%
%:%294=97%:%
%:%296=99%:%
%:%297=99%:%
%:%312=102%:%
%:%316=103%:%
%:%320=104%:%
%:%324=105%:%
%:%326=107%:%
%:%327=107%:%
%:%341=110%:%
%:%343=112%:%
%:%346=114%:%
%:%347=115%:%
%:%348=116%:%
%:%349=117%:%
%:%350=118%:%
%:%351=119%:%
%:%355=121%:%
%:%357=123%:%
%:%358=123%:%
%:%373=126%:%
%:%374=127%:%
%:%375=128%:%
%:%376=129%:%
%:%377=130%:%
%:%378=131%:%
%:%379=132%:%
%:%381=134%:%
%:%382=134%:%
%:%397=138%:%
%:%398=139%:%
%:%399=140%:%
%:%400=141%:%
%:%401=142%:%
%:%402=143%:%
%:%403=144%:%
%:%405=146%:%
%:%406=146%:%
%:%421=149%:%
%:%422=150%:%
%:%424=152%:%
%:%425=152%:%
%:%439=155%:%
