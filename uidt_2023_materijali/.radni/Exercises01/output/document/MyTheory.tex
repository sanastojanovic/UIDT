%
\begin{isabellebody}%
\setisabellecontext{MyTheory}%
%
\isadelimtheory
%
\endisadelimtheory
%
\isatagtheory
%
\endisatagtheory
{\isafoldtheory}%
%
\isadelimtheory
%
\endisadelimtheory
%
\begin{exercise}[subtitle=Primer jednostavne teorije]
%
\begin{isamarkuptext}%
(a) Pokazati da važi komutativnost i asocijativnost 
          operacije \isa{{\isacharparenleft}{\kern0pt}{\isacharplus}{\kern0pt}{\isacharparenright}{\kern0pt}\ {\isacharcolon}{\kern0pt}{\isacharcolon}{\kern0pt}\ nat\ {\isasymRightarrow}\ nat\ {\isasymRightarrow}\ nat}.%
\end{isamarkuptext}\isamarkuptrue%
\isacommand{lemma}\isamarkupfalse%
\ {\isachardoublequoteopen}{\isacharparenleft}{\kern0pt}x{\isacharcolon}{\kern0pt}{\isacharcolon}{\kern0pt}nat{\isacharparenright}{\kern0pt}\ {\isacharplus}{\kern0pt}\ y\ {\isacharequal}{\kern0pt}\ y\ {\isacharplus}{\kern0pt}\ x{\isachardoublequoteclose}\isanewline
%
\isadelimproof
\ \ %
\endisadelimproof
%
\isatagproof
\isacommand{by}\isamarkupfalse%
\ simp%
\endisatagproof
{\isafoldproof}%
%
\isadelimproof
\isanewline
%
\endisadelimproof
\isanewline
\isacommand{lemma}\isamarkupfalse%
\ {\isachardoublequoteopen}{\isacharparenleft}{\kern0pt}{\isacharparenleft}{\kern0pt}x{\isacharcolon}{\kern0pt}{\isacharcolon}{\kern0pt}nat{\isacharparenright}{\kern0pt}\ {\isacharplus}{\kern0pt}\ y{\isacharparenright}{\kern0pt}\ {\isacharplus}{\kern0pt}\ z\ {\isacharequal}{\kern0pt}\ x\ {\isacharplus}{\kern0pt}\ {\isacharparenleft}{\kern0pt}y\ {\isacharplus}{\kern0pt}\ z{\isacharparenright}{\kern0pt}{\isachardoublequoteclose}\isanewline
%
\isadelimproof
\ \ %
\endisadelimproof
%
\isatagproof
\isacommand{by}\isamarkupfalse%
\ simp%
\endisatagproof
{\isafoldproof}%
%
\isadelimproof
%
\endisadelimproof
%
\begin{isamarkuptext}%
(b) Definisati funkciju \isa{sledbenik\ {\isacharcolon}{\kern0pt}{\isacharcolon}{\kern0pt}\ nat\ {\isasymRightarrow}\ nat} i 
          pokazati da važi \isa{sledbenik\ {\isacharparenleft}{\kern0pt}sledbenik\ x{\isacharparenright}{\kern0pt}\ {\isacharequal}{\kern0pt}\ x\ {\isacharplus}{\kern0pt}\ {\isadigit{2}}}.%
\end{isamarkuptext}\isamarkuptrue%
\isacommand{definition}\isamarkupfalse%
\ sledbenik\ {\isacharcolon}{\kern0pt}{\isacharcolon}{\kern0pt}\ {\isachardoublequoteopen}nat\ {\isasymRightarrow}\ nat{\isachardoublequoteclose}\ \isakeyword{where}\isanewline
\ \ {\isachardoublequoteopen}sledbenik\ x\ {\isacharequal}{\kern0pt}\ x\ {\isacharplus}{\kern0pt}\ {\isadigit{1}}{\isachardoublequoteclose}\isanewline
\isanewline
\isacommand{lemma}\isamarkupfalse%
\ {\isachardoublequoteopen}sledbenik\ {\isacharparenleft}{\kern0pt}sledbenik\ x{\isacharparenright}{\kern0pt}\ {\isacharequal}{\kern0pt}\ x\ {\isacharplus}{\kern0pt}\ {\isadigit{2}}{\isachardoublequoteclose}\isanewline
%
\isadelimproof
\ \ %
\endisadelimproof
%
\isatagproof
\isacommand{unfolding}\isamarkupfalse%
\ sledbenik{\isacharunderscore}{\kern0pt}def\isanewline
\ \ \isacommand{by}\isamarkupfalse%
\ simp%
\endisatagproof
{\isafoldproof}%
%
\isadelimproof
%
\endisadelimproof
%
\begin{isamarkuptext}%
(c) Pokazati da ako važi \isa{x\ {\isachargreater}{\kern0pt}\ {\isadigit{0}}} onda \isa{sledbenik\ x\ {\isachargreater}{\kern0pt}\ {\isadigit{1}}}.
          Te pokazati da ako važi \isa{x\ {\isacharless}{\kern0pt}\ {\isadigit{5}}} onda \isa{sledbenik\ x\ {\isacharless}{\kern0pt}\ {\isadigit{6}}}.%
\end{isamarkuptext}\isamarkuptrue%
\isacommand{lemma}\isamarkupfalse%
\ {\isachardoublequoteopen}x\ {\isachargreater}{\kern0pt}\ {\isadigit{0}}\ {\isasymlongrightarrow}\ sledbenik\ x\ {\isachargreater}{\kern0pt}\ {\isadigit{1}}{\isachardoublequoteclose}\isanewline
%
\isadelimproof
\ \ %
\endisadelimproof
%
\isatagproof
\isacommand{unfolding}\isamarkupfalse%
\ sledbenik{\isacharunderscore}{\kern0pt}def\isanewline
\ \ \isacommand{by}\isamarkupfalse%
\ simp%
\endisatagproof
{\isafoldproof}%
%
\isadelimproof
\isanewline
%
\endisadelimproof
\isanewline
\isacommand{lemma}\isamarkupfalse%
\ {\isachardoublequoteopen}x\ {\isacharless}{\kern0pt}\ {\isadigit{5}}\ {\isasymlongrightarrow}\ sledbenik\ x\ {\isacharless}{\kern0pt}\ {\isadigit{6}}{\isachardoublequoteclose}\isanewline
%
\isadelimproof
\ \ %
\endisadelimproof
%
\isatagproof
\isacommand{unfolding}\isamarkupfalse%
\ sledbenik{\isacharunderscore}{\kern0pt}def\isanewline
\ \ \isacommand{by}\isamarkupfalse%
\ simp%
\endisatagproof
{\isafoldproof}%
%
\isadelimproof
%
\endisadelimproof
%
\begin{isamarkuptext}%
(d) Prethodna dva tvrđenja uopštiti u opšte tvrđenje o ograničenosti sledbenika.%
\end{isamarkuptext}\isamarkuptrue%
\isacommand{lemma}\isamarkupfalse%
\ ogranicenost{\isacharunderscore}{\kern0pt}sledbenika{\isacharcolon}{\kern0pt}\isanewline
\ \ \isakeyword{fixes}\ a\ b\ {\isacharcolon}{\kern0pt}{\isacharcolon}{\kern0pt}\ nat\isanewline
\ \ \isakeyword{assumes}\ {\isachardoublequoteopen}a\ {\isacharless}{\kern0pt}\ x{\isachardoublequoteclose}\ {\isachardoublequoteopen}x\ {\isacharless}{\kern0pt}\ b{\isachardoublequoteclose}\isanewline
\ \ \isakeyword{shows}\ {\isachardoublequoteopen}a\ {\isacharplus}{\kern0pt}\ {\isadigit{1}}\ {\isacharless}{\kern0pt}\ sledbenik\ x\ {\isasymand}\ sledbenik\ x\ {\isacharless}{\kern0pt}\ b\ {\isacharplus}{\kern0pt}\ {\isadigit{1}}{\isachardoublequoteclose}\isanewline
%
\isadelimproof
\ \ %
\endisadelimproof
%
\isatagproof
\isacommand{unfolding}\isamarkupfalse%
\ sledbenik{\isacharunderscore}{\kern0pt}def\isanewline
\ \ \isacommand{using}\isamarkupfalse%
\ assms\isanewline
\ \ \isacommand{by}\isamarkupfalse%
\ simp%
\endisatagproof
{\isafoldproof}%
%
\isadelimproof
%
\endisadelimproof
%
\begin{isamarkuptext}%
(e) Definisati funkciju \isa{kvadrat\ {\isacharcolon}{\kern0pt}{\isacharcolon}{\kern0pt}\ nat\ {\isasymRightarrow}\ nat} i
          pokazati da važi \isa{kvadrat\ {\isacharparenleft}{\kern0pt}x\ {\isacharplus}{\kern0pt}\ {\isadigit{1}}{\isacharparenright}{\kern0pt}\ {\isacharequal}{\kern0pt}\ kvadrat\ x\ {\isacharplus}{\kern0pt}\ {\isadigit{2}}\ {\isacharasterisk}{\kern0pt}\ x\ {\isacharplus}{\kern0pt}\ {\isadigit{1}}}.%
\end{isamarkuptext}\isamarkuptrue%
\isacommand{abbreviation}\isamarkupfalse%
\ kvadrat\ {\isacharcolon}{\kern0pt}{\isacharcolon}{\kern0pt}\ {\isachardoublequoteopen}nat\ {\isasymRightarrow}\ nat{\isachardoublequoteclose}\ \isakeyword{where}\isanewline
\ \ {\isachardoublequoteopen}kvadrat\ x\ {\isasymequiv}\ x\ {\isacharasterisk}{\kern0pt}\ x{\isachardoublequoteclose}\isanewline
\isanewline
\isacommand{lemma}\isamarkupfalse%
\ {\isachardoublequoteopen}kvadrat\ {\isacharparenleft}{\kern0pt}x\ {\isacharplus}{\kern0pt}\ {\isadigit{1}}{\isacharparenright}{\kern0pt}\ {\isacharequal}{\kern0pt}\ kvadrat\ x\ {\isacharplus}{\kern0pt}\ {\isadigit{2}}\ {\isacharasterisk}{\kern0pt}\ x\ {\isacharplus}{\kern0pt}\ {\isadigit{1}}{\isachardoublequoteclose}\isanewline
%
\isadelimproof
\ \ %
\endisadelimproof
%
\isatagproof
\isacommand{by}\isamarkupfalse%
\ simp%
\endisatagproof
{\isafoldproof}%
%
\isadelimproof
%
\endisadelimproof
%
\begin{isamarkuptext}%
(f) Definisati rekurzivnu funkciju \isa{sum\ {\isacharcolon}{\kern0pt}{\isacharcolon}{\kern0pt}\ nat\ list\ {\isasymRightarrow}\ nat} koja računa sumu 
          liste prirodnih brojeva. Pokazati da se \isa{sum\ xs} ponaša isto kao 
          i \isa{foldr} primenjen na odgovarajuću funkciju, listu \isa{xs}, i 
          odgovarajuću početnu vrenodst akomulatora. Nako toga pokazati sledeće svojstvo 
          \isa{sum\ {\isacharparenleft}{\kern0pt}xs\ {\isacharat}{\kern0pt}\ ys{\isacharparenright}{\kern0pt}\ {\isacharequal}{\kern0pt}\ sum\ xs\ {\isacharplus}{\kern0pt}\ sum\ ys}.%
\end{isamarkuptext}\isamarkuptrue%
\isacommand{fun}\isamarkupfalse%
\ sum\ {\isacharcolon}{\kern0pt}{\isacharcolon}{\kern0pt}\ {\isachardoublequoteopen}nat\ list\ {\isasymRightarrow}\ nat{\isachardoublequoteclose}\ \isakeyword{where}\isanewline
\ \ {\isachardoublequoteopen}sum\ {\isacharbrackleft}{\kern0pt}{\isacharbrackright}{\kern0pt}\ {\isacharequal}{\kern0pt}\ {\isadigit{0}}{\isachardoublequoteclose}\isanewline
{\isacharbar}{\kern0pt}\ {\isachardoublequoteopen}sum\ {\isacharparenleft}{\kern0pt}x\ {\isacharhash}{\kern0pt}\ xs{\isacharparenright}{\kern0pt}\ {\isacharequal}{\kern0pt}\ x\ {\isacharplus}{\kern0pt}\ sum\ xs{\isachardoublequoteclose}\isanewline
\isanewline
\isacommand{lemma}\isamarkupfalse%
\ {\isachardoublequoteopen}sum\ xs\ {\isacharequal}{\kern0pt}\ foldr\ {\isacharparenleft}{\kern0pt}{\isacharplus}{\kern0pt}{\isacharparenright}{\kern0pt}\ xs\ {\isadigit{0}}{\isachardoublequoteclose}\isanewline
%
\isadelimproof
\ \ %
\endisadelimproof
%
\isatagproof
\isacommand{by}\isamarkupfalse%
\ {\isacharparenleft}{\kern0pt}induction\ xs{\isacharparenright}{\kern0pt}\ auto%
\endisatagproof
{\isafoldproof}%
%
\isadelimproof
\isanewline
%
\endisadelimproof
\isanewline
\isacommand{lemma}\isamarkupfalse%
\ {\isachardoublequoteopen}sum\ {\isacharparenleft}{\kern0pt}xs\ {\isacharat}{\kern0pt}\ ys{\isacharparenright}{\kern0pt}\ {\isacharequal}{\kern0pt}\ sum\ xs\ {\isacharplus}{\kern0pt}\ sum\ ys{\isachardoublequoteclose}\isanewline
%
\isadelimproof
\ \ %
\endisadelimproof
%
\isatagproof
\isacommand{by}\isamarkupfalse%
\ {\isacharparenleft}{\kern0pt}induction\ xs{\isacharparenright}{\kern0pt}\ auto%
\endisatagproof
{\isafoldproof}%
%
\isadelimproof
%
\endisadelimproof
%
\begin{isamarkuptext}%
(g) Definisati rekurzivnu funkciju \isa{len\ {\isacharcolon}{\kern0pt}{\isacharcolon}{\kern0pt}\ nat\ list\ {\isasymRightarrow}\ nat} koja računa dužinu 
          liste prirodnih brojeva. Pokazati da se \isa{len\ xs} ponaša isto kao 
          i \isa{foldr} primenjen na odgovarajuću funkciju, listu \isa{xs}, i
          odgovarajuću početnu vrednost akumulatora (Savet: Zgodno je koristiti 
          lambda funkciju \isa{{\isacharparenleft}{\kern0pt}{\isasymlambda}\ x\ y{\isachardot}{\kern0pt}\ f\ x\ y{\isacharparenright}{\kern0pt}} za definisanje funkcije koju prima
          \isa{foldr}). Nako toga pokazati sledeće svojstvo 
          \isa{len\ {\isacharparenleft}{\kern0pt}xs\ {\isacharat}{\kern0pt}\ ys{\isacharparenright}{\kern0pt}\ {\isacharequal}{\kern0pt}\ len\ xs\ {\isacharplus}{\kern0pt}\ len\ ys}.%
\end{isamarkuptext}\isamarkuptrue%
\isacommand{fun}\isamarkupfalse%
\ len\ {\isacharcolon}{\kern0pt}{\isacharcolon}{\kern0pt}\ {\isachardoublequoteopen}nat\ list\ {\isasymRightarrow}\ nat{\isachardoublequoteclose}\ \isakeyword{where}\isanewline
\ \ {\isachardoublequoteopen}len\ {\isacharbrackleft}{\kern0pt}{\isacharbrackright}{\kern0pt}\ {\isacharequal}{\kern0pt}\ {\isadigit{0}}{\isachardoublequoteclose}\isanewline
{\isacharbar}{\kern0pt}\ {\isachardoublequoteopen}len\ {\isacharparenleft}{\kern0pt}x\ {\isacharhash}{\kern0pt}\ xs{\isacharparenright}{\kern0pt}\ {\isacharequal}{\kern0pt}\ {\isadigit{1}}\ {\isacharplus}{\kern0pt}\ len\ xs{\isachardoublequoteclose}\isanewline
\isanewline
\isacommand{lemma}\isamarkupfalse%
\ {\isachardoublequoteopen}len\ xs\ {\isacharequal}{\kern0pt}\ foldr\ {\isacharparenleft}{\kern0pt}{\isasymlambda}\ x{\isachardot}{\kern0pt}\ {\isacharparenleft}{\kern0pt}{\isacharplus}{\kern0pt}{\isacharparenright}{\kern0pt}\ {\isadigit{1}}{\isacharparenright}{\kern0pt}\ xs\ {\isadigit{0}}{\isachardoublequoteclose}\isanewline
%
\isadelimproof
\ \ %
\endisadelimproof
%
\isatagproof
\isacommand{by}\isamarkupfalse%
\ {\isacharparenleft}{\kern0pt}induction\ xs{\isacharparenright}{\kern0pt}\ auto%
\endisatagproof
{\isafoldproof}%
%
\isadelimproof
\isanewline
%
\endisadelimproof
\isanewline
\isacommand{lemma}\isamarkupfalse%
\ {\isachardoublequoteopen}len\ {\isacharparenleft}{\kern0pt}xs\ {\isacharat}{\kern0pt}\ ys{\isacharparenright}{\kern0pt}\ {\isacharequal}{\kern0pt}\ len\ xs\ {\isacharplus}{\kern0pt}\ len\ ys{\isachardoublequoteclose}\isanewline
%
\isadelimproof
\ \ %
\endisadelimproof
%
\isatagproof
\isacommand{by}\isamarkupfalse%
\ {\isacharparenleft}{\kern0pt}induction\ xs{\isacharparenright}{\kern0pt}\ auto%
\endisatagproof
{\isafoldproof}%
%
\isadelimproof
%
\endisadelimproof
%
\end{exercise}
%
\begin{exercise}[subtitle=Zapisivanje logičkih formula]
%
\begin{isamarkuptext}%
(a) Zapisati nekoliko logičkih formula koje uključuju 
          logičke konstante \isa{True} i \isa{False},
          logičke veznike \isa{{\isasymnot}}, \isa{{\isasymand}}, \isa{{\isasymor}}, 
          \isa{{\isasymlongrightarrow}}, i \isa{{\isasymlongleftrightarrow}}/\isa{{\isacharequal}{\kern0pt}}, i
          univerzalne i egzistencionalne kvantifikatore \isa{{\isasymforall}} i \isa{{\isasymexists}}%
\end{isamarkuptext}\isamarkuptrue%
\isacommand{lemma}\isamarkupfalse%
\ {\isachardoublequoteopen}A\ {\isasymand}\ B\ {\isasymlongrightarrow}\ A\ {\isasymor}\ B{\isachardoublequoteclose}\isanewline
%
\isadelimproof
\ \ %
\endisadelimproof
%
\isatagproof
\isacommand{by}\isamarkupfalse%
\ simp%
\endisatagproof
{\isafoldproof}%
%
\isadelimproof
\isanewline
%
\endisadelimproof
\isanewline
\isacommand{lemma}\isamarkupfalse%
\ {\isachardoublequoteopen}A\ {\isasymand}\ A\ {\isasymlongleftrightarrow}\ A{\isachardoublequoteclose}\isanewline
%
\isadelimproof
\ \ %
\endisadelimproof
%
\isatagproof
\isacommand{by}\isamarkupfalse%
\ simp%
\endisatagproof
{\isafoldproof}%
%
\isadelimproof
\isanewline
%
\endisadelimproof
\isanewline
\isacommand{lemma}\isamarkupfalse%
\ {\isachardoublequoteopen}A\ {\isasymor}\ {\isasymnot}\ A\ {\isasymlongleftrightarrow}\ True{\isachardoublequoteclose}\isanewline
%
\isadelimproof
\ \ %
\endisadelimproof
%
\isatagproof
\isacommand{by}\isamarkupfalse%
\ simp%
\endisatagproof
{\isafoldproof}%
%
\isadelimproof
\isanewline
%
\endisadelimproof
\isanewline
\isacommand{lemma}\isamarkupfalse%
\ {\isachardoublequoteopen}{\isasymforall}\ x{\isachardot}{\kern0pt}\ P\ x\ {\isasymlongrightarrow}\ Q\ x{\isachardoublequoteclose}\isanewline
\ \ \isacommand{nitpick}\isamarkupfalse%
\isanewline
%
\isadelimproof
\ \ %
\endisadelimproof
%
\isatagproof
\isacommand{oops}\isamarkupfalse%
%
\endisatagproof
{\isafoldproof}%
%
\isadelimproof
\isanewline
%
\endisadelimproof
\isanewline
\isacommand{lemma}\isamarkupfalse%
\ {\isachardoublequoteopen}{\isacharparenleft}{\kern0pt}{\isasymforall}\ x{\isachardot}{\kern0pt}\ P\ x\ {\isasymlongrightarrow}\ Q\ x{\isacharparenright}{\kern0pt}\ {\isasymand}\ {\isacharparenleft}{\kern0pt}{\isasymexists}\ x{\isachardot}{\kern0pt}\ P\ x{\isacharparenright}{\kern0pt}\ {\isasymlongrightarrow}\ {\isacharparenleft}{\kern0pt}{\isasymexists}\ x{\isachardot}{\kern0pt}\ Q\ x{\isacharparenright}{\kern0pt}{\isachardoublequoteclose}\isanewline
\ \ \isacommand{sledgehammer}\isamarkupfalse%
\isanewline
%
\isadelimproof
\ \ %
\endisadelimproof
%
\isatagproof
\isacommand{by}\isamarkupfalse%
\ blast%
\endisatagproof
{\isafoldproof}%
%
\isadelimproof
%
\endisadelimproof
%
\begin{isamarkuptext}%
(b) Zapisati sledeće rečenice u logici prvog reda i dokazati njihovu ispravnost.%
\end{isamarkuptext}\isamarkuptrue%
%
\begin{isamarkuptext}%
(b.1) Ako onaj ko laže taj i krade i ako bar neko laže, onda neko i krade.%
\end{isamarkuptext}\isamarkuptrue%
\isacommand{lemma}\isamarkupfalse%
\ {\isachardoublequoteopen}\isanewline
\ \ \ \ {\isacharparenleft}{\kern0pt}{\isasymforall}\ x{\isachardot}{\kern0pt}\ Laze\ x\ {\isasymlongrightarrow}\ Krade\ x{\isacharparenright}{\kern0pt}\ {\isasymand}\isanewline
\ \ \ \ {\isacharparenleft}{\kern0pt}{\isasymexists}\ x{\isachardot}{\kern0pt}\ Laze\ x{\isacharparenright}{\kern0pt}\ {\isasymlongrightarrow}\isanewline
\ \ \ \ {\isacharparenleft}{\kern0pt}{\isasymexists}\ x{\isachardot}{\kern0pt}\ Krade\ x{\isacharparenright}{\kern0pt}{\isachardoublequoteclose}\isanewline
%
\isadelimproof
\ \ %
\endisadelimproof
%
\isatagproof
\isacommand{by}\isamarkupfalse%
\ auto%
\endisatagproof
{\isafoldproof}%
%
\isadelimproof
%
\endisadelimproof
%
\begin{isamarkuptext}%
(b.2) Ako ”ko radi taj ima ili troši” i ”ko ima taj peva” i ”ko troši taj peva”, onda
”ko radi taj peva”%
\end{isamarkuptext}\isamarkuptrue%
\isacommand{lemma}\isamarkupfalse%
\ {\isachardoublequoteopen}\isanewline
\ \ \ \ {\isacharparenleft}{\kern0pt}{\isasymforall}\ x{\isachardot}{\kern0pt}\ Radi\ x\ {\isasymlongrightarrow}\ Ima\ x\ {\isasymand}\ Trosi\ x{\isacharparenright}{\kern0pt}\ {\isasymand}\isanewline
\ \ \ \ {\isacharparenleft}{\kern0pt}{\isasymforall}\ x{\isachardot}{\kern0pt}\ Ima\ x\ {\isasymlongrightarrow}\ Peva\ x{\isacharparenright}{\kern0pt}\ {\isasymand}\isanewline
\ \ \ \ {\isacharparenleft}{\kern0pt}{\isasymforall}\ x{\isachardot}{\kern0pt}\ Trosi\ x\ {\isasymlongrightarrow}\ Peva\ x{\isacharparenright}{\kern0pt}\ {\isasymlongrightarrow}\isanewline
\ \ \ \ {\isacharparenleft}{\kern0pt}{\isasymforall}\ x{\isachardot}{\kern0pt}\ Radi\ x\ {\isasymlongrightarrow}\ Peva\ x{\isacharparenright}{\kern0pt}{\isachardoublequoteclose}\isanewline
%
\isadelimproof
\ \ %
\endisadelimproof
%
\isatagproof
\isacommand{by}\isamarkupfalse%
\ auto%
\endisatagproof
{\isafoldproof}%
%
\isadelimproof
%
\endisadelimproof
%
\begin{isamarkuptext}%
(c) Zapisati sledeći skup rečenica u logici prvog reda i dokazati njihovu 
          nezadovoljivost.%
\end{isamarkuptext}\isamarkuptrue%
%
\begin{isamarkuptext}%
(c.1) Ako je X prijatelj osobe Y, onda je i Y prijatelj osobe X.%
\end{isamarkuptext}\isamarkuptrue%
%
\begin{isamarkuptext}%
(c.2) Ako je X prijatelj osobe Y, onda X voli Y.%
\end{isamarkuptext}\isamarkuptrue%
%
\begin{isamarkuptext}%
(c.3) Ne postoji neko ko je povredio osobu koju voli.%
\end{isamarkuptext}\isamarkuptrue%
%
\begin{isamarkuptext}%
(c.4) Osoba Y je povredila svog prijatelja X.%
\end{isamarkuptext}\isamarkuptrue%
\isacommand{lemma}\isamarkupfalse%
\ {\isachardoublequoteopen}\isanewline
\ \ \ \ {\isacharparenleft}{\kern0pt}{\isasymforall}\ x\ y{\isachardot}{\kern0pt}\ Prijetelj\ x\ y\ {\isasymlongrightarrow}\ Prijatelj\ y\ x{\isacharparenright}{\kern0pt}\ {\isasymand}\isanewline
\ \ \ \ {\isacharparenleft}{\kern0pt}{\isasymforall}\ x\ y{\isachardot}{\kern0pt}\ Prijatelj\ x\ y\ {\isasymlongrightarrow}\ Voli\ x\ y{\isacharparenright}{\kern0pt}\ {\isasymand}\isanewline
\ \ \ \ {\isacharparenleft}{\kern0pt}{\isasymnot}\ {\isacharparenleft}{\kern0pt}{\isasymexists}\ x\ y{\isachardot}{\kern0pt}\ Voli\ x\ y\ {\isasymand}\ Povredio\ x\ y{\isacharparenright}{\kern0pt}{\isacharparenright}{\kern0pt}\ {\isasymand}\isanewline
\ \ \ \ {\isacharparenleft}{\kern0pt}{\isasymexists}\ y\ x{\isachardot}{\kern0pt}\ Prijatelj\ y\ x\ {\isasymand}\ Povredio\ y\ x{\isacharparenright}{\kern0pt}\ {\isasymlongrightarrow}\isanewline
\ \ \ \ False{\isachardoublequoteclose}\isanewline
%
\isadelimproof
\ \ %
\endisadelimproof
%
\isatagproof
\isacommand{by}\isamarkupfalse%
\ auto%
\endisatagproof
{\isafoldproof}%
%
\isadelimproof
%
\endisadelimproof
%
\end{exercise}
%
\isadelimtheory
%
\endisadelimtheory
%
\isatagtheory
%
\endisatagtheory
{\isafoldtheory}%
%
\isadelimtheory
%
\endisadelimtheory
%
\end{isabellebody}%
\endinput
%:%file=MyTheory.tex%:%
%:%18=8%:%
%:%21=10%:%
%:%22=11%:%
%:%24=13%:%
%:%25=13%:%
%:%28=14%:%
%:%32=14%:%
%:%33=14%:%
%:%38=14%:%
%:%41=15%:%
%:%42=16%:%
%:%43=16%:%
%:%46=17%:%
%:%50=17%:%
%:%51=17%:%
%:%60=19%:%
%:%61=20%:%
%:%63=22%:%
%:%64=22%:%
%:%65=23%:%
%:%66=24%:%
%:%67=25%:%
%:%68=25%:%
%:%71=26%:%
%:%75=26%:%
%:%76=26%:%
%:%77=27%:%
%:%78=27%:%
%:%87=29%:%
%:%88=30%:%
%:%90=32%:%
%:%91=32%:%
%:%94=33%:%
%:%98=33%:%
%:%99=33%:%
%:%100=34%:%
%:%101=34%:%
%:%106=34%:%
%:%109=35%:%
%:%110=36%:%
%:%111=36%:%
%:%114=37%:%
%:%118=37%:%
%:%119=37%:%
%:%120=38%:%
%:%121=38%:%
%:%130=40%:%
%:%132=42%:%
%:%133=42%:%
%:%134=43%:%
%:%135=44%:%
%:%136=45%:%
%:%139=46%:%
%:%143=46%:%
%:%144=46%:%
%:%145=47%:%
%:%146=47%:%
%:%147=48%:%
%:%148=48%:%
%:%157=50%:%
%:%158=51%:%
%:%160=53%:%
%:%161=53%:%
%:%162=54%:%
%:%163=55%:%
%:%164=56%:%
%:%165=56%:%
%:%168=57%:%
%:%172=57%:%
%:%173=57%:%
%:%182=59%:%
%:%183=60%:%
%:%184=61%:%
%:%185=62%:%
%:%186=63%:%
%:%188=65%:%
%:%189=65%:%
%:%190=66%:%
%:%191=67%:%
%:%192=68%:%
%:%193=69%:%
%:%194=69%:%
%:%197=70%:%
%:%201=70%:%
%:%202=70%:%
%:%207=70%:%
%:%210=71%:%
%:%211=72%:%
%:%212=72%:%
%:%215=73%:%
%:%219=73%:%
%:%220=73%:%
%:%229=75%:%
%:%230=76%:%
%:%231=77%:%
%:%232=78%:%
%:%233=79%:%
%:%234=80%:%
%:%235=81%:%
%:%237=83%:%
%:%238=83%:%
%:%239=84%:%
%:%240=85%:%
%:%241=86%:%
%:%242=87%:%
%:%243=87%:%
%:%246=88%:%
%:%250=88%:%
%:%251=88%:%
%:%256=88%:%
%:%259=89%:%
%:%260=90%:%
%:%261=90%:%
%:%264=91%:%
%:%268=91%:%
%:%269=91%:%
%:%277=93%:%
%:%279=95%:%
%:%282=97%:%
%:%283=98%:%
%:%284=99%:%
%:%285=100%:%
%:%286=101%:%
%:%288=103%:%
%:%289=103%:%
%:%292=104%:%
%:%296=104%:%
%:%297=104%:%
%:%302=104%:%
%:%305=105%:%
%:%306=106%:%
%:%307=106%:%
%:%310=107%:%
%:%314=107%:%
%:%315=107%:%
%:%320=107%:%
%:%323=108%:%
%:%324=109%:%
%:%325=109%:%
%:%328=110%:%
%:%332=110%:%
%:%333=110%:%
%:%338=110%:%
%:%341=111%:%
%:%342=112%:%
%:%343=112%:%
%:%344=113%:%
%:%345=113%:%
%:%348=114%:%
%:%352=114%:%
%:%358=114%:%
%:%361=115%:%
%:%362=116%:%
%:%363=116%:%
%:%364=117%:%
%:%365=117%:%
%:%368=118%:%
%:%372=118%:%
%:%373=118%:%
%:%382=120%:%
%:%386=122%:%
%:%388=124%:%
%:%389=124%:%
%:%392=127%:%
%:%395=128%:%
%:%399=128%:%
%:%400=128%:%
%:%409=130%:%
%:%410=131%:%
%:%412=133%:%
%:%413=133%:%
%:%417=137%:%
%:%420=138%:%
%:%424=138%:%
%:%425=138%:%
%:%434=140%:%
%:%435=141%:%
%:%439=143%:%
%:%443=144%:%
%:%447=145%:%
%:%451=146%:%
%:%453=148%:%
%:%454=148%:%
%:%459=153%:%
%:%462=154%:%
%:%466=154%:%
%:%467=154%:%
%:%475=156%:%
