\documentclass[11pt,a4paper]{article}
\usepackage[utf8]{inputenc}
\usepackage[serbian]{babel}
\usepackage[margin=1in]{geometry}
\usepackage{isabelle,isabellesym}
\usepackage{amsthm}
\usepackage{amsmath}
\usepackage{amsfonts}
\usepackage{amssymb}
\usepackage{graphicx}
\usepackage{xsim}

\xsimsetup{%
    exercise/name=Zadatak,
    solution/name=Rešenje,
    exercises/name=Zadaci,
    solutions/name=Rešenja
}

\theoremstyle{plain}% default
\newtheorem{thm}{Teorema}[section]
\newtheorem{lem}[thm]{Lema}
\newtheorem{prop}[thm]{Proposition}
\newtheorem*{cor}{Posledica}
\theoremstyle{definition}
\newtheorem{defn}{Definicija}[section]
\newtheorem{exmp}{Primer}[section]
\newtheorem{xca}[exmp]{Zadatak}
\theoremstyle{remark}
\newtheorem*{rem}{Primedba}
\newtheorem*{note}{Opaska}
\newtheorem{case}{Slučaj}

% further packages required for unusual symbols (see also
% isabellesym.sty), use only when needed

\usepackage{amssymb}
  %for \<leadsto>, \<box>, \<diamond>, \<sqsupset>, \<mho>, \<Join>,
  %\<lhd>, \<lesssim>, \<greatersim>, \<lessapprox>, \<greaterapprox>,
  %\<triangleq>, \<yen>, \<lozenge>

\usepackage{eurosym}
  %for \<euro>

\usepackage[only,bigsqcap,bigparallel,fatsemi,interleave,sslash]{stmaryrd}
  %for \<Sqinter>, \<Parallel>, \<Zsemi>, \<Parallel>, \<sslash>

\usepackage{eufrak}
  %for \<AA> ... \<ZZ>, \<aa> ... \<zz> (also included in amssymb)

\usepackage{textcomp}
  %for \<onequarter>, \<onehalf>, \<threequarters>, \<degree>, \<cent>,
  %\<currency>

% this should be the last package used
\usepackage{pdfsetup}

% urls in roman style, theory text in math-similar italics
\urlstyle{rm}
\isabellestyle{it}

% for uniform font size
\renewcommand{\isastyle}{\isastyleminor}


\begin{document}

% sane default for proof documents
\parindent 0pt\parskip 0.5ex

\begin{tabular*}{\textwidth}{c@{\extracolsep{\fill}}r}
    Univerzitet u Beogradu & Šifra predmeta: R265\\
    Matematički fakultet & 21.02.2023.\\
    Katedra za računarstvo i informatiku &
\end{tabular*}

\begin{center}
    \begin{huge}
        Uvod u interaktivno dokazivanje teorema
    \end{huge}

    \vspace{5pt}

    \begin{Large}
        Vežbe 1
    \end{Large}
\end{center}


% generated text of all theories
%
\begin{isabellebody}%
\setisabellecontext{MyTheory}%
%
\isadelimtheory
%
\endisadelimtheory
%
\isatagtheory
%
\endisatagtheory
{\isafoldtheory}%
%
\isadelimtheory
%
\endisadelimtheory
%
\begin{exercise}[subtitle=Tip: list.]
%
\begin{isamarkuptext}%
Diskutovati o sledećim termovima i vrednostima.%
\end{isamarkuptext}\isamarkuptrue%
\isacommand{term}\isamarkupfalse%
\ {\isachardoublequoteopen}{\isacharbrackleft}{\kern0pt}{\isacharbrackright}{\kern0pt}{\isachardoublequoteclose}\isanewline
\isacommand{term}\isamarkupfalse%
\ {\isachardoublequoteopen}{\isadigit{1}}\ {\isacharhash}{\kern0pt}\ {\isadigit{2}}\ {\isacharhash}{\kern0pt}\ {\isacharbrackleft}{\kern0pt}{\isacharbrackright}{\kern0pt}{\isachardoublequoteclose}\isanewline
\isacommand{term}\isamarkupfalse%
\ {\isachardoublequoteopen}{\isacharparenleft}{\kern0pt}{\isadigit{1}}{\isacharcolon}{\kern0pt}{\isacharcolon}{\kern0pt}nat{\isacharparenright}{\kern0pt}\ {\isacharhash}{\kern0pt}\ {\isadigit{2}}\ {\isacharhash}{\kern0pt}\ {\isacharbrackleft}{\kern0pt}{\isacharbrackright}{\kern0pt}{\isachardoublequoteclose}\isanewline
\isacommand{term}\isamarkupfalse%
\ {\isachardoublequoteopen}{\isacharbrackleft}{\kern0pt}{\isadigit{1}}{\isacharcomma}{\kern0pt}\ {\isadigit{2}}{\isacharbrackright}{\kern0pt}{\isachardoublequoteclose}\isanewline
\isacommand{term}\isamarkupfalse%
\ {\isachardoublequoteopen}{\isacharbrackleft}{\kern0pt}{\isadigit{1}}{\isacharcolon}{\kern0pt}{\isacharcolon}{\kern0pt}nat{\isacharcomma}{\kern0pt}\ {\isadigit{2}}{\isacharbrackright}{\kern0pt}{\isachardoublequoteclose}\isanewline
\isanewline
\isacommand{value}\isamarkupfalse%
\ {\isachardoublequoteopen}{\isacharbrackleft}{\kern0pt}{\isadigit{1}}{\isachardot}{\kern0pt}{\isachardot}{\kern0pt}{\isadigit{5}}{\isacharbrackright}{\kern0pt}{\isachardoublequoteclose}\isanewline
\isacommand{value}\isamarkupfalse%
\ {\isachardoublequoteopen}{\isacharbrackleft}{\kern0pt}{\isadigit{1}}{\isachardot}{\kern0pt}{\isachardot}{\kern0pt}{\isacharless}{\kern0pt}{\isadigit{5}}{\isacharbrackright}{\kern0pt}{\isachardoublequoteclose}\isanewline
\isanewline
\isacommand{term}\isamarkupfalse%
\ sum{\isacharunderscore}{\kern0pt}list\isanewline
\isacommand{value}\isamarkupfalse%
\ {\isachardoublequoteopen}sum{\isacharunderscore}{\kern0pt}list\ {\isacharbrackleft}{\kern0pt}{\isadigit{1}}{\isachardot}{\kern0pt}{\isachardot}{\kern0pt}{\isacharless}{\kern0pt}{\isadigit{5}}{\isacharbrackright}{\kern0pt}{\isachardoublequoteclose}\isanewline
\isanewline
\isacommand{term}\isamarkupfalse%
\ map\isanewline
\isacommand{term}\isamarkupfalse%
\ {\isachardoublequoteopen}{\isasymlambda}\ x{\isachardot}{\kern0pt}\ f\ x{\isachardoublequoteclose}\isanewline
\isacommand{value}\isamarkupfalse%
\ {\isachardoublequoteopen}map\ {\isacharparenleft}{\kern0pt}{\isasymlambda}\ x{\isachardot}{\kern0pt}\ x{\isacharcircum}{\kern0pt}{\isadigit{2}}{\isacharparenright}{\kern0pt}\ {\isacharbrackleft}{\kern0pt}{\isadigit{1}}{\isachardot}{\kern0pt}{\isachardot}{\kern0pt}{\isacharless}{\kern0pt}{\isadigit{5}}{\isacharbrackright}{\kern0pt}{\isachardoublequoteclose}\isanewline
\isacommand{value}\isamarkupfalse%
\ {\isachardoublequoteopen}sum{\isacharunderscore}{\kern0pt}list\ {\isacharparenleft}{\kern0pt}map\ {\isacharparenleft}{\kern0pt}{\isasymlambda}\ x{\isachardot}{\kern0pt}\ x{\isacharcircum}{\kern0pt}{\isadigit{2}}{\isacharparenright}{\kern0pt}\ {\isacharbrackleft}{\kern0pt}{\isadigit{1}}{\isachardot}{\kern0pt}{\isachardot}{\kern0pt}{\isacharless}{\kern0pt}{\isadigit{5}}{\isacharbrackright}{\kern0pt}{\isacharparenright}{\kern0pt}{\isachardoublequoteclose}\isanewline
\isanewline
\isacommand{value}\isamarkupfalse%
\ {\isachardoublequoteopen}{\isasymSum}\ x\ {\isasymleftarrow}\ {\isacharbrackleft}{\kern0pt}{\isadigit{1}}{\isachardot}{\kern0pt}{\isachardot}{\kern0pt}{\isacharless}{\kern0pt}{\isadigit{5}}{\isacharbrackright}{\kern0pt}{\isachardot}{\kern0pt}\ x{\isacharcircum}{\kern0pt}{\isadigit{2}}{\isachardoublequoteclose}%
\end{exercise}
%
\begin{exercise}[subtitle=Sumiranje nizova preko listi.]
%
\begin{isamarkuptext}%
Pokazati da važi: $1 + 2^2 + \ldots + n^2 = \frac{n (n + 1) (2n + 1)}{6}$.%
\end{isamarkuptext}\isamarkuptrue%
\isacommand{primrec}\isamarkupfalse%
\ zbir{\isacharunderscore}{\kern0pt}kvadrata\ {\isacharcolon}{\kern0pt}{\isacharcolon}{\kern0pt}\ {\isachardoublequoteopen}nat\ {\isasymRightarrow}\ nat{\isachardoublequoteclose}\ \isakeyword{where}\isanewline
\ \ {\isachardoublequoteopen}zbir{\isacharunderscore}{\kern0pt}kvadrata\ {\isadigit{0}}\ {\isacharequal}{\kern0pt}\ {\isadigit{0}}{\isachardoublequoteclose}\isanewline
{\isacharbar}{\kern0pt}\ {\isachardoublequoteopen}zbir{\isacharunderscore}{\kern0pt}kvadrata\ {\isacharparenleft}{\kern0pt}Suc\ n{\isacharparenright}{\kern0pt}\ {\isacharequal}{\kern0pt}\ zbir{\isacharunderscore}{\kern0pt}kvadrata\ n\ {\isacharplus}{\kern0pt}\ {\isacharparenleft}{\kern0pt}Suc\ n{\isacharparenright}{\kern0pt}\ {\isacharcircum}{\kern0pt}\ {\isadigit{2}}{\isachardoublequoteclose}%
\begin{isamarkuptext}%
Definisati funkciju \isa{zbir{\isacharunderscore}{\kern0pt}kvadrata{\isacharprime}{\kern0pt}\ {\isacharcolon}{\kern0pt}{\isacharcolon}{\kern0pt}\ nat\ {\isasymRightarrow}\ nat} preko definicije,
      koja računa levu stranu jednakosti pomoću liste i funkcijama nad listama.%
\end{isamarkuptext}\isamarkuptrue%
\isacommand{definition}\isamarkupfalse%
\ zbir{\isacharunderscore}{\kern0pt}kvadrata{\isacharprime}{\kern0pt}\ {\isacharcolon}{\kern0pt}{\isacharcolon}{\kern0pt}\ {\isachardoublequoteopen}nat\ {\isasymRightarrow}\ nat{\isachardoublequoteclose}\ \isakeyword{where}\isanewline
\ \ {\isachardoublequoteopen}zbir{\isacharunderscore}{\kern0pt}kvadrata{\isacharprime}{\kern0pt}\ n\ {\isacharequal}{\kern0pt}\ undefined{\isachardoublequoteclose}%
\begin{isamarkuptext}%
Pokazati da su ove dve funkcije ekvivalentne.%
\end{isamarkuptext}\isamarkuptrue%
\isacommand{lemma}\isamarkupfalse%
\ {\isachardoublequoteopen}zbir{\isacharunderscore}{\kern0pt}kvadrata\ n\ {\isacharequal}{\kern0pt}\ zbir{\isacharunderscore}{\kern0pt}kvadrata{\isacharprime}{\kern0pt}\ n{\isachardoublequoteclose}\isanewline
\ \ %
\isadelimproof
%
\endisadelimproof
%
\isatagproof
%
\endisatagproof
{\isafoldproof}%
%
\isadelimproof
%
\endisadelimproof
%
\begin{isamarkuptext}%
Pokazati automatski da je \isa{zbir{\isacharunderscore}{\kern0pt}kvadrata\ n\ {\isacharequal}{\kern0pt}\ n\ {\isacharasterisk}{\kern0pt}\ {\isacharparenleft}{\kern0pt}n\ {\isacharplus}{\kern0pt}\ {\isadigit{1}}{\isacharparenright}{\kern0pt}\ {\isacharasterisk}{\kern0pt}\ {\isacharparenleft}{\kern0pt}{\isadigit{2}}\ {\isacharasterisk}{\kern0pt}\ n\ {\isacharplus}{\kern0pt}\ {\isadigit{1}}{\isacharparenright}{\kern0pt}\ div\ {\isadigit{6}}}.\\
      \isa{Savet}: Razmotriti leme koje se koriste u Isar verziji dokaza i dodati ih u \isa{simp}.%
\end{isamarkuptext}\isamarkuptrue%
\isacommand{lemma}\isamarkupfalse%
\ {\isachardoublequoteopen}zbir{\isacharunderscore}{\kern0pt}kvadrata\ n\ {\isacharequal}{\kern0pt}\ n\ {\isacharasterisk}{\kern0pt}\ {\isacharparenleft}{\kern0pt}n\ {\isacharplus}{\kern0pt}\ {\isadigit{1}}{\isacharparenright}{\kern0pt}\ {\isacharasterisk}{\kern0pt}\ {\isacharparenleft}{\kern0pt}{\isadigit{2}}\ {\isacharasterisk}{\kern0pt}\ n\ {\isacharplus}{\kern0pt}\ {\isadigit{1}}{\isacharparenright}{\kern0pt}\ div\ {\isadigit{6}}{\isachardoublequoteclose}\isanewline
\ \ %
\isadelimproof
%
\endisadelimproof
%
\isatagproof
%
\endisatagproof
{\isafoldproof}%
%
\isadelimproof
%
\endisadelimproof
%
\end{exercise}
%
\begin{exercise}[subtitle=Algebarski tip podataka: lista.]
%
\begin{isamarkuptext}%
Definisati polimorfan algebarski tip podataka \isa{{\isacharprime}{\kern0pt}a\ lista}
      koji predstavlja listu elemenata polimorfong tipa \isa{{\isacharprime}{\kern0pt}a}.%
\end{isamarkuptext}\isamarkuptrue%
\isacommand{datatype}\isamarkupfalse%
\ {\isacharprime}{\kern0pt}a\ lista\ {\isacharequal}{\kern0pt}\ Prazna\isanewline
\ \ \ \ \ \ \ \ \ \ \ \ \ \ \ \ \ \ {\isacharbar}{\kern0pt}\ Dodaj\ {\isacharprime}{\kern0pt}a\ {\isachardoublequoteopen}{\isacharprime}{\kern0pt}a\ lista{\isachardoublequoteclose}\isanewline
\isanewline
\isacommand{term}\isamarkupfalse%
\ {\isachardoublequoteopen}Dodaj\ {\isacharparenleft}{\kern0pt}{\isadigit{1}}{\isacharcolon}{\kern0pt}{\isacharcolon}{\kern0pt}nat{\isacharparenright}{\kern0pt}\ {\isacharparenleft}{\kern0pt}Dodaj\ {\isadigit{2}}\ {\isacharparenleft}{\kern0pt}Dodaj\ {\isadigit{3}}\ Prazna{\isacharparenright}{\kern0pt}{\isacharparenright}{\kern0pt}{\isachardoublequoteclose}%
\begin{isamarkuptext}%
Definisati funkcije 
      \isa{duzina{\isacharprime}{\kern0pt}\ {\isacharcolon}{\kern0pt}{\isacharcolon}{\kern0pt}\ {\isacharprime}{\kern0pt}a\ lista\ {\isasymRightarrow}\ nat}, 
      \isa{nadovezi{\isacharprime}{\kern0pt}\ {\isacharcolon}{\kern0pt}{\isacharcolon}{\kern0pt}\ {\isacharprime}{\kern0pt}a\ lista\ {\isasymRightarrow}\ {\isacharprime}{\kern0pt}a\ lista\ {\isasymRightarrow}\ {\isacharprime}{\kern0pt}a\ lista},
      \isa{obrni{\isacharprime}{\kern0pt}\ {\isacharcolon}{\kern0pt}{\isacharcolon}{\kern0pt}\ {\isacharprime}{\kern0pt}a\ lista\ {\isasymRightarrow}\ {\isacharprime}{\kern0pt}a\ lista}
      primitivnom rekurzijom koje računaju
      dužinu liste, nadoveziju i obrću liste tipa \isa{{\isacharprime}{\kern0pt}a\ lista}.%
\end{isamarkuptext}\isamarkuptrue%
%
\begin{isamarkuptext}%
Definisati funkciju \isa{duzina\ {\isacharcolon}{\kern0pt}{\isacharcolon}{\kern0pt}\ {\isacharprime}{\kern0pt}a\ list\ {\isasymRightarrow}\ nat} primitivnom rekurzijom 
      koja računa dužinu liste tipa \isa{{\isacharprime}{\kern0pt}a\ list}.
      Ta pokazati da su \isa{duzina} i \isa{length} ekvivalentne funkcije.%
\end{isamarkuptext}\isamarkuptrue%
\isacommand{primrec}\isamarkupfalse%
\ duzina\ {\isacharcolon}{\kern0pt}{\isacharcolon}{\kern0pt}\ {\isachardoublequoteopen}{\isacharprime}{\kern0pt}a\ list\ {\isasymRightarrow}\ nat{\isachardoublequoteclose}\ \isakeyword{where}\isanewline
\ \ {\isachardoublequoteopen}duzina\ {\isacharbrackleft}{\kern0pt}{\isacharbrackright}{\kern0pt}\ {\isacharequal}{\kern0pt}\ undefined{\isachardoublequoteclose}\isanewline
{\isacharbar}{\kern0pt}\ {\isachardoublequoteopen}duzina\ {\isacharparenleft}{\kern0pt}x\ {\isacharhash}{\kern0pt}\ xs{\isacharparenright}{\kern0pt}\ {\isacharequal}{\kern0pt}\ undefined{\isachardoublequoteclose}\isanewline
\isanewline
\isacommand{lemma}\isamarkupfalse%
\ duzina{\isacharunderscore}{\kern0pt}length{\isacharcolon}{\kern0pt}\isanewline
\ \ \isakeyword{shows}\ {\isachardoublequoteopen}duzina\ xs\ {\isacharequal}{\kern0pt}\ length\ xs{\isachardoublequoteclose}\isanewline
\ \ %
\isadelimproof
%
\endisadelimproof
%
\isatagproof
%
\endisatagproof
{\isafoldproof}%
%
\isadelimproof
%
\endisadelimproof
%
\begin{isamarkuptext}%
Definisati funkciju \isa{prebroj\ {\isacharcolon}{\kern0pt}{\isacharcolon}{\kern0pt}\ {\isacharparenleft}{\kern0pt}{\isacharprime}{\kern0pt}a{\isacharcolon}{\kern0pt}{\isacharcolon}{\kern0pt}equal{\isacharparenright}{\kern0pt}\ {\isasymRightarrow}\ {\isacharprime}{\kern0pt}a\ list\ {\isasymRightarrow}\ nat} primitivnom rekurzijom 
      koja računa koliko se puta javlja element tipa \isa{{\isacharprime}{\kern0pt}a{\isacharcolon}{\kern0pt}{\isacharcolon}{\kern0pt}equal} u listi tipa \isa{{\isacharparenleft}{\kern0pt}{\isacharprime}{\kern0pt}a{\isacharcolon}{\kern0pt}{\isacharcolon}{\kern0pt}equal{\isacharparenright}{\kern0pt}\ list}. 
      Ta pokazati da je \isa{prebroj\ a\ xs\ {\isasymle}\ length\ xs}.%
\end{isamarkuptext}\isamarkuptrue%
%
\begin{isamarkuptext}%
Definisati funkicju \isa{sadrzi\ {\isacharcolon}{\kern0pt}{\isacharcolon}{\kern0pt}\ {\isacharparenleft}{\kern0pt}{\isacharprime}{\kern0pt}a{\isacharcolon}{\kern0pt}{\isacharcolon}{\kern0pt}equal{\isacharparenright}{\kern0pt}\ {\isasymRightarrow}\ {\isacharprime}{\kern0pt}a\ list\ {\isasymRightarrow}\ bool} primitivnom rekurzijom
      koja ispituje da li se element tipa \isa{{\isacharprime}{\kern0pt}a{\isacharcolon}{\kern0pt}{\isacharcolon}{\kern0pt}equal} javlja u listi tipa \isa{{\isacharparenleft}{\kern0pt}{\isacharprime}{\kern0pt}a{\isacharcolon}{\kern0pt}{\isacharcolon}{\kern0pt}equal{\isacharparenright}{\kern0pt}\ list}.
      Ta pokazati da je \isa{sadrzi\ a\ xs\ {\isacharequal}{\kern0pt}\ a\ {\isasymin}\ set\ xs}%
\end{isamarkuptext}\isamarkuptrue%
%
\begin{isamarkuptext}%
Definisati funkciju \isa{skup\ {\isacharcolon}{\kern0pt}{\isacharcolon}{\kern0pt}\ {\isacharprime}{\kern0pt}a\ list\ {\isasymRightarrow}\ {\isacharprime}{\kern0pt}a\ set} primitivnom rekurzijom
      koja vraća skup tipa \isa{{\isacharprime}{\kern0pt}a\ set} koji je sačinjen od elemenata liste tipa \isa{{\isacharprime}{\kern0pt}a\ list}.
      Ta pokazati da je \isa{skup\ xs\ {\isacharequal}{\kern0pt}\ set\ xs}.%
\end{isamarkuptext}\isamarkuptrue%
\isacommand{primrec}\isamarkupfalse%
\ skup\ {\isacharcolon}{\kern0pt}{\isacharcolon}{\kern0pt}\ {\isachardoublequoteopen}{\isacharprime}{\kern0pt}a\ list\ {\isasymRightarrow}\ {\isacharprime}{\kern0pt}a\ set{\isachardoublequoteclose}\ \isakeyword{where}\isanewline
\ \ {\isachardoublequoteopen}skup\ {\isacharbrackleft}{\kern0pt}{\isacharbrackright}{\kern0pt}\ {\isacharequal}{\kern0pt}\ undefined{\isachardoublequoteclose}\isanewline
{\isacharbar}{\kern0pt}\ {\isachardoublequoteopen}skup\ {\isacharparenleft}{\kern0pt}x\ {\isacharhash}{\kern0pt}\ xs{\isacharparenright}{\kern0pt}\ {\isacharequal}{\kern0pt}\ undefined{\isachardoublequoteclose}\isanewline
\isanewline
\isacommand{lemma}\isamarkupfalse%
\ skup{\isacharunderscore}{\kern0pt}set{\isacharcolon}{\kern0pt}\isanewline
\ \ \isakeyword{shows}\ {\isachardoublequoteopen}skup\ xs\ {\isacharequal}{\kern0pt}\ set\ xs{\isachardoublequoteclose}\isanewline
\ \ %
\isadelimproof
%
\endisadelimproof
%
\isatagproof
%
\endisatagproof
{\isafoldproof}%
%
\isadelimproof
%
\endisadelimproof
%
\begin{isamarkuptext}%
Definisati funkciju \isa{nadovezi\ {\isacharcolon}{\kern0pt}{\isacharcolon}{\kern0pt}\ {\isacharprime}{\kern0pt}a\ list\ {\isasymRightarrow}\ {\isacharprime}{\kern0pt}a\ list\ {\isasymRightarrow}\ {\isacharprime}{\kern0pt}a\ list} primitivnom rekurzijom
      koja nadovezuje jednu listu na drugu tipa \isa{{\isacharprime}{\kern0pt}a\ list}.
      Ta pokazati da je ekvivalentna ugrađenoj funkciji \isa{append} 
      ili infiksom operatoru \isa{{\isacharat}{\kern0pt}}.%
\end{isamarkuptext}\isamarkuptrue%
\isacommand{primrec}\isamarkupfalse%
\ nadovezi\ {\isacharcolon}{\kern0pt}{\isacharcolon}{\kern0pt}\ {\isachardoublequoteopen}{\isacharprime}{\kern0pt}a\ list\ {\isasymRightarrow}\ {\isacharprime}{\kern0pt}a\ list\ {\isasymRightarrow}\ {\isacharprime}{\kern0pt}a\ list{\isachardoublequoteclose}\ \isakeyword{where}\isanewline
\ \ {\isachardoublequoteopen}nadovezi\ {\isacharbrackleft}{\kern0pt}{\isacharbrackright}{\kern0pt}\ {\isacharequal}{\kern0pt}\ undefined{\isachardoublequoteclose}\isanewline
{\isacharbar}{\kern0pt}\ {\isachardoublequoteopen}nadovezi\ {\isacharparenleft}{\kern0pt}x\ {\isacharhash}{\kern0pt}\ xs{\isacharparenright}{\kern0pt}\ {\isacharequal}{\kern0pt}\ undefined{\isachardoublequoteclose}%
\begin{isamarkuptext}%
Formulisati i pokazati da je dužina dve nedovezane liste, zbir dužina pojedinačnih listi.\\
      Orediti i dokazati osobine za funkcije \isa{skup} i \isa{nadovezi}, kao i za \isa{sadrzi} i \isa{nadovezi}.%
\end{isamarkuptext}\isamarkuptrue%
%
\begin{isamarkuptext}%
Definisati funkicju \isa{obrni\ {\isacharcolon}{\kern0pt}{\isacharcolon}{\kern0pt}\ {\isacharprime}{\kern0pt}a\ list\ {\isasymRightarrow}\ {\isacharprime}{\kern0pt}a\ list} primitivnom rekurzijom
      koja obrće listu tipa \isa{{\isacharprime}{\kern0pt}a\ list}. 
      Ta pokazati da funkcija je \isa{obrni} ekvivalentna funkciji \isa{rev}.
      Nakon toga pokazati da je dvostruko obrnuta lista
      ekvivalentna početnoj listi.\\
      \isa{Napomena}: Pri definisanju funkcije \isa{obrni} nije dozvoljeno 
                  koristiti operator nadovezivanje \isa{{\isacharat}{\kern0pt}}.\\
      \isa{Savet}: Potrebno je definisati pomoćne leme.%
\end{isamarkuptext}\isamarkuptrue%
\isacommand{primrec}\isamarkupfalse%
\ obrni\ {\isacharcolon}{\kern0pt}{\isacharcolon}{\kern0pt}\ {\isachardoublequoteopen}{\isacharprime}{\kern0pt}a\ list\ {\isasymRightarrow}\ {\isacharprime}{\kern0pt}a\ list{\isachardoublequoteclose}\ \isakeyword{where}\isanewline
\ \ {\isachardoublequoteopen}obrni\ {\isacharbrackleft}{\kern0pt}{\isacharbrackright}{\kern0pt}\ {\isacharequal}{\kern0pt}\ undefined{\isachardoublequoteclose}\isanewline
{\isacharbar}{\kern0pt}\ {\isachardoublequoteopen}obrni\ {\isacharparenleft}{\kern0pt}x\ {\isacharhash}{\kern0pt}\ xs{\isacharparenright}{\kern0pt}\ {\isacharequal}{\kern0pt}\ undefined{\isachardoublequoteclose}\isanewline
\isanewline
\isacommand{lemma}\isamarkupfalse%
\ obrni{\isacharunderscore}{\kern0pt}rev{\isacharcolon}{\kern0pt}\ \isanewline
\ \ \isakeyword{shows}\ {\isachardoublequoteopen}obrni\ xs\ {\isacharequal}{\kern0pt}\ rev\ xs{\isachardoublequoteclose}\isanewline
\ \ %
\isadelimproof
%
\endisadelimproof
%
\isatagproof
%
\endisatagproof
{\isafoldproof}%
%
\isadelimproof
%
\endisadelimproof
\isanewline
\isacommand{lemma}\isamarkupfalse%
\ obrni{\isacharunderscore}{\kern0pt}obrni{\isacharunderscore}{\kern0pt}id{\isacharcolon}{\kern0pt}\ {\isachardoublequoteopen}obrni\ {\isacharparenleft}{\kern0pt}obrni\ xs{\isacharparenright}{\kern0pt}\ {\isacharequal}{\kern0pt}\ xs{\isachardoublequoteclose}\isanewline
\ \ %
\isadelimproof
%
\endisadelimproof
%
\isatagproof
%
\endisatagproof
{\isafoldproof}%
%
\isadelimproof
%
\endisadelimproof
%
\begin{isamarkuptext}%
Definisati funkciju \isa{snoc\ {\isacharcolon}{\kern0pt}{\isacharcolon}{\kern0pt}\ {\isacharprime}{\kern0pt}a\ {\isasymRightarrow}\ {\isacharprime}{\kern0pt}a\ list\ {\isasymRightarrow}\ {\isacharprime}{\kern0pt}a\ list} koja dodaje element 
      na kraj liste, i funkciju \isa{rev{\isacharunderscore}{\kern0pt}snoc\ {\isacharcolon}{\kern0pt}{\isacharcolon}{\kern0pt}\ {\isacharprime}{\kern0pt}a\ list\ {\isasymRightarrow}\ {\isacharprime}{\kern0pt}a\ list} koja uz pomoć 
      funkcije \isa{snoc} obrće elemente liste. Da li \isa{rev{\isacharunderscore}{\kern0pt}snoc} popravlja složenost
      obrtanja liste?%
\end{isamarkuptext}\isamarkuptrue%
\isacommand{primrec}\isamarkupfalse%
\ snoc\ {\isacharcolon}{\kern0pt}{\isacharcolon}{\kern0pt}\ {\isachardoublequoteopen}{\isacharprime}{\kern0pt}a\ {\isasymRightarrow}\ {\isacharprime}{\kern0pt}a\ list\ {\isasymRightarrow}\ {\isacharprime}{\kern0pt}a\ list{\isachardoublequoteclose}\ \isakeyword{where}\isanewline
\ \ {\isachardoublequoteopen}snoc\ a\ {\isacharbrackleft}{\kern0pt}{\isacharbrackright}{\kern0pt}\ {\isacharequal}{\kern0pt}\ undefined{\isachardoublequoteclose}\isanewline
{\isacharbar}{\kern0pt}\ {\isachardoublequoteopen}snoc\ a\ {\isacharparenleft}{\kern0pt}x\ {\isacharhash}{\kern0pt}\ xs{\isacharparenright}{\kern0pt}\ {\isacharequal}{\kern0pt}\ undefined{\isachardoublequoteclose}\isanewline
\isanewline
\isacommand{primrec}\isamarkupfalse%
\ rev{\isacharunderscore}{\kern0pt}snoc\ {\isacharcolon}{\kern0pt}{\isacharcolon}{\kern0pt}\ {\isachardoublequoteopen}{\isacharprime}{\kern0pt}a\ list\ {\isasymRightarrow}\ {\isacharprime}{\kern0pt}a\ list{\isachardoublequoteclose}\ \isakeyword{where}\isanewline
\ \ {\isachardoublequoteopen}rev{\isacharunderscore}{\kern0pt}snoc\ {\isacharbrackleft}{\kern0pt}{\isacharbrackright}{\kern0pt}\ {\isacharequal}{\kern0pt}\ undefined{\isachardoublequoteclose}\isanewline
{\isacharbar}{\kern0pt}\ {\isachardoublequoteopen}rev{\isacharunderscore}{\kern0pt}snoc\ {\isacharparenleft}{\kern0pt}x\ {\isacharhash}{\kern0pt}\ xs{\isacharparenright}{\kern0pt}\ {\isacharequal}{\kern0pt}\ undefined{\isachardoublequoteclose}%
\begin{isamarkuptext}%
Definisati funkciju \isa{itrev} koja obrće listu iterativno.\\
      \isa{Savet}: Koristiti pomoćnu listu.%
\end{isamarkuptext}\isamarkuptrue%
%
\begin{isamarkuptext}%
Pokazati da je funkcija \isa{itrev} ekvivalentna ugrađenoj
      funkciji \isa{rev}, kada je inicijalna pomoćna lista prazna.%
\end{isamarkuptext}\isamarkuptrue%
%
\begin{isamarkuptext}%
Pomoću funkcije \isa{fold} opisati obrtanje liste.
      Pokazati ekvivalentnost funkciji \isa{itrev} sa obrtanjem liste preko \isa{fold}-a.%
\end{isamarkuptext}\isamarkuptrue%
\isacommand{term}\isamarkupfalse%
\ fold%
\end{exercise}
%
\isadelimtheory
%
\endisadelimtheory
%
\isatagtheory
%
\endisatagtheory
{\isafoldtheory}%
%
\isadelimtheory
%
\endisadelimtheory
%
\end{isabellebody}%
\endinput
%:%file=MyTheory.tex%:%
%:%18=8%:%
%:%21=10%:%
%:%23=12%:%
%:%24=12%:%
%:%25=13%:%
%:%26=13%:%
%:%27=14%:%
%:%28=14%:%
%:%29=15%:%
%:%30=15%:%
%:%31=16%:%
%:%32=16%:%
%:%33=17%:%
%:%34=18%:%
%:%35=18%:%
%:%36=19%:%
%:%37=19%:%
%:%38=20%:%
%:%39=21%:%
%:%40=21%:%
%:%41=22%:%
%:%42=22%:%
%:%43=23%:%
%:%44=24%:%
%:%45=24%:%
%:%46=25%:%
%:%47=25%:%
%:%48=26%:%
%:%49=26%:%
%:%50=27%:%
%:%51=27%:%
%:%52=28%:%
%:%53=29%:%
%:%54=29%:%
%:%55=31%:%
%:%57=33%:%
%:%60=35%:%
%:%62=37%:%
%:%63=37%:%
%:%64=38%:%
%:%65=39%:%
%:%67=41%:%
%:%68=42%:%
%:%70=44%:%
%:%71=44%:%
%:%72=45%:%
%:%74=47%:%
%:%76=49%:%
%:%77=49%:%
%:%78=50%:%
%:%93=52%:%
%:%94=53%:%
%:%96=55%:%
%:%97=55%:%
%:%98=56%:%
%:%112=58%:%
%:%114=60%:%
%:%117=62%:%
%:%118=63%:%
%:%120=65%:%
%:%121=65%:%
%:%122=66%:%
%:%123=67%:%
%:%124=68%:%
%:%125=68%:%
%:%127=70%:%
%:%128=71%:%
%:%129=72%:%
%:%130=73%:%
%:%131=74%:%
%:%132=75%:%
%:%136=77%:%
%:%137=78%:%
%:%138=79%:%
%:%140=81%:%
%:%141=81%:%
%:%142=82%:%
%:%143=83%:%
%:%144=84%:%
%:%145=85%:%
%:%146=85%:%
%:%147=86%:%
%:%148=87%:%
%:%163=89%:%
%:%164=90%:%
%:%165=91%:%
%:%169=93%:%
%:%170=94%:%
%:%171=95%:%
%:%175=97%:%
%:%176=98%:%
%:%177=99%:%
%:%179=101%:%
%:%180=101%:%
%:%181=102%:%
%:%182=103%:%
%:%183=104%:%
%:%184=105%:%
%:%185=105%:%
%:%186=106%:%
%:%187=107%:%
%:%202=109%:%
%:%203=110%:%
%:%204=111%:%
%:%205=112%:%
%:%207=114%:%
%:%208=114%:%
%:%209=115%:%
%:%210=116%:%
%:%212=118%:%
%:%213=119%:%
%:%217=121%:%
%:%218=122%:%
%:%219=123%:%
%:%220=124%:%
%:%221=125%:%
%:%222=126%:%
%:%223=127%:%
%:%224=128%:%
%:%226=130%:%
%:%227=130%:%
%:%228=131%:%
%:%229=132%:%
%:%230=133%:%
%:%231=134%:%
%:%232=134%:%
%:%233=135%:%
%:%234=136%:%
%:%247=137%:%
%:%248=138%:%
%:%249=138%:%
%:%250=139%:%
%:%265=141%:%
%:%266=142%:%
%:%267=143%:%
%:%268=144%:%
%:%270=146%:%
%:%271=146%:%
%:%272=147%:%
%:%273=148%:%
%:%274=149%:%
%:%275=150%:%
%:%276=150%:%
%:%277=151%:%
%:%278=152%:%
%:%280=154%:%
%:%281=155%:%
%:%285=157%:%
%:%286=158%:%
%:%290=160%:%
%:%291=161%:%
%:%293=163%:%
%:%294=163%:%
%:%295=165%:%



% optional bibliography
%\bibliographystyle{abbrv}
%\bibliography{root}

\end{document}

%%% Local Variables:
%%% mode: latex
%%% TeX-master: t
%%% End:
