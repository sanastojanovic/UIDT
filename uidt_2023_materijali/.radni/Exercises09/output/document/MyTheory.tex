%
\begin{isabellebody}%
\setisabellecontext{MyTheory}%
%
\isadelimtheory
%
\endisadelimtheory
%
\isatagtheory
%
\endisatagtheory
{\isafoldtheory}%
%
\isadelimtheory
%
\endisadelimtheory
%
\begin{exercise}[subtitle=Zasnivanje prirodnih brojeva.]
%
\begin{isamarkuptext}%
Definisati algebarski tip podataka \isa{prirodni} koji predstavlja prirodni broj.%
\end{isamarkuptext}\isamarkuptrue%
\isacommand{datatype}\isamarkupfalse%
\ prirodni\ {\isacharequal}{\kern0pt}\ undef%
\begin{isamarkuptext}%
Diskutovati o tipu \isa{prirodni} i sledećim termovima.%
\end{isamarkuptext}\isamarkuptrue%
\isacommand{typ}\isamarkupfalse%
\ prirodni\isanewline
\isanewline
\isacommand{term}\isamarkupfalse%
\ {\isachardoublequoteopen}Nula{\isachardoublequoteclose}\isanewline
\isacommand{term}\isamarkupfalse%
\ {\isachardoublequoteopen}Sled\ Nula{\isachardoublequoteclose}\isanewline
\isacommand{term}\isamarkupfalse%
\ {\isachardoublequoteopen}Sled\ {\isacharparenleft}{\kern0pt}Sled\ Nula{\isacharparenright}{\kern0pt}{\isachardoublequoteclose}%
\begin{isamarkuptext}%
Definisati skraćenice za prirodne brojeve \isa{{\isasymone}{\isacharcomma}{\kern0pt}\ {\isasymtwo}{\isacharcomma}{\kern0pt}\ {\isasymthree}}.%
\end{isamarkuptext}\isamarkuptrue%
\isacommand{abbreviation}\isamarkupfalse%
\ jedan\ {\isacharcolon}{\kern0pt}{\isacharcolon}{\kern0pt}\ prirodni\ {\isacharparenleft}{\kern0pt}{\isachardoublequoteopen}{\isasymone}{\isachardoublequoteclose}{\isacharparenright}{\kern0pt}\ \isakeyword{where}\isanewline
\ \ {\isachardoublequoteopen}{\isasymone}\ {\isasymequiv}\ undefined{\isachardoublequoteclose}\isanewline
\isanewline
\isacommand{abbreviation}\isamarkupfalse%
\ dva\ {\isacharcolon}{\kern0pt}{\isacharcolon}{\kern0pt}\ prirodni\ {\isacharparenleft}{\kern0pt}{\isachardoublequoteopen}{\isasymtwo}{\isachardoublequoteclose}{\isacharparenright}{\kern0pt}\ \isakeyword{where}\isanewline
\ \ {\isachardoublequoteopen}{\isasymtwo}\ {\isasymequiv}\ undefined{\isachardoublequoteclose}\isanewline
\isanewline
\isacommand{abbreviation}\isamarkupfalse%
\ tri\ {\isacharcolon}{\kern0pt}{\isacharcolon}{\kern0pt}\ prirodni\ {\isacharparenleft}{\kern0pt}{\isachardoublequoteopen}{\isasymthree}{\isachardoublequoteclose}{\isacharparenright}{\kern0pt}\ \isakeyword{where}\isanewline
\ \ {\isachardoublequoteopen}{\isasymthree}\ {\isasymequiv}\ undefined{\isachardoublequoteclose}%
\begin{isamarkuptext}%
Primitivnom rekurzijom definisati operaciju sabiranja. Uvesti levo 
      asocijativni operator \isa{{\isasymoplus}} za operaciju sabiranja.%
\end{isamarkuptext}\isamarkuptrue%
\isacommand{fun}\isamarkupfalse%
\ saberi\ {\isacharparenleft}{\kern0pt}\isakeyword{infixl}\ {\isachardoublequoteopen}{\isasymoplus}{\isachardoublequoteclose}\ {\isadigit{1}}{\isadigit{0}}{\isadigit{0}}{\isacharparenright}{\kern0pt}\ \isakeyword{where}\isanewline
\ \ {\isachardoublequoteopen}a\ {\isasymoplus}\ b\ {\isacharequal}{\kern0pt}\ undefined{\isachardoublequoteclose}%
\begin{isamarkuptext}%
Testirati funkciju sabiranjem nekih skraćenica za prirodne brojeve.%
\end{isamarkuptext}\isamarkuptrue%
%
\begin{isamarkuptext}%
Pokazati da je sabiranje asocijativno.%
\end{isamarkuptext}\isamarkuptrue%
\isacommand{lemma}\isamarkupfalse%
\ saberi{\isacharunderscore}{\kern0pt}asoc{\isacharcolon}{\kern0pt}\isanewline
\ \ \isakeyword{shows}\ {\isachardoublequoteopen}a\ {\isasymoplus}\ {\isacharparenleft}{\kern0pt}b\ {\isasymoplus}\ c{\isacharparenright}{\kern0pt}\ {\isacharequal}{\kern0pt}\ a\ {\isasymoplus}\ b\ {\isasymoplus}\ c{\isachardoublequoteclose}\isanewline
\ \ %
\isadelimproof
%
\endisadelimproof
%
\isatagproof
%
\endisatagproof
{\isafoldproof}%
%
\isadelimproof
%
\endisadelimproof
%
\begin{isamarkuptext}%
Pokazati da je sabiranje komutativno.\\
     \isa{Savet}: Potrebno je pokazati pomoćne lemu.%
\end{isamarkuptext}\isamarkuptrue%
\isacommand{lemma}\isamarkupfalse%
\ saberi{\isacharunderscore}{\kern0pt}kom{\isacharcolon}{\kern0pt}\isanewline
\ \ \isakeyword{shows}\ {\isachardoublequoteopen}a\ {\isasymoplus}\ b\ {\isacharequal}{\kern0pt}\ b\ {\isasymoplus}\ a{\isachardoublequoteclose}\isanewline
\ \ %
\isadelimproof
%
\endisadelimproof
%
\isatagproof
%
\endisatagproof
{\isafoldproof}%
%
\isadelimproof
%
\endisadelimproof
\isanewline
\isacommand{lemma}\isamarkupfalse%
\ saberi{\isacharunderscore}{\kern0pt}kom{\isacharunderscore}{\kern0pt}isar{\isacharcolon}{\kern0pt}\isanewline
\ \ \isakeyword{shows}\ {\isachardoublequoteopen}a\ {\isasymoplus}\ b\ {\isacharequal}{\kern0pt}\ b\ {\isasymoplus}\ a{\isachardoublequoteclose}\isanewline
\ \ %
\isadelimproof
%
\endisadelimproof
%
\isatagproof
%
\endisatagproof
{\isafoldproof}%
%
\isadelimproof
%
\endisadelimproof
%
\begin{isamarkuptext}%
Primitivnom rekurzijom definisati operaciju množenja. Uvesti levo 
      asocijativni operator \isa{{\isasymotimes}} za operaciju množenja.%
\end{isamarkuptext}\isamarkuptrue%
\isacommand{fun}\isamarkupfalse%
\ pomnozi\ {\isacharparenleft}{\kern0pt}\isakeyword{infixl}\ {\isachardoublequoteopen}{\isasymotimes}{\isachardoublequoteclose}\ {\isadigit{1}}{\isadigit{0}}{\isadigit{1}}{\isacharparenright}{\kern0pt}\ \isakeyword{where}\isanewline
\ \ {\isachardoublequoteopen}a\ {\isasymotimes}\ b\ {\isacharequal}{\kern0pt}\ undefined{\isachardoublequoteclose}%
\begin{isamarkuptext}%
Pokazati komutativnost množenja.\\
     \isa{Savet}: Pokazati pomoćne lemme.%
\end{isamarkuptext}\isamarkuptrue%
\isacommand{lemma}\isamarkupfalse%
\ pomnozi{\isacharunderscore}{\kern0pt}kom{\isacharcolon}{\kern0pt}\isanewline
\ \ \isakeyword{shows}\ {\isachardoublequoteopen}a\ {\isasymotimes}\ b\ {\isacharequal}{\kern0pt}\ b\ {\isasymotimes}\ a{\isachardoublequoteclose}\isanewline
\ \ %
\isadelimproof
%
\endisadelimproof
%
\isatagproof
%
\endisatagproof
{\isafoldproof}%
%
\isadelimproof
%
\endisadelimproof
%
\begin{isamarkuptext}%
Pokazati da je množenje asocijativno.%
\end{isamarkuptext}\isamarkuptrue%
\isacommand{lemma}\isamarkupfalse%
\ pomnozi{\isacharunderscore}{\kern0pt}asoc{\isacharcolon}{\kern0pt}\isanewline
\ \ \isakeyword{shows}\ {\isachardoublequoteopen}a\ {\isasymotimes}\ {\isacharparenleft}{\kern0pt}b\ {\isasymotimes}\ c{\isacharparenright}{\kern0pt}\ {\isacharequal}{\kern0pt}\ a\ {\isasymotimes}\ b\ {\isasymotimes}\ c{\isachardoublequoteclose}\isanewline
\ \ %
\isadelimproof
%
\endisadelimproof
%
\isatagproof
%
\endisatagproof
{\isafoldproof}%
%
\isadelimproof
%
\endisadelimproof
%
\begin{isamarkuptext}%
Primitivnom rekurzijom definisati operaciju stepenovanja. Uvesti desno 
      asocijativni operator \isa{{\isasymZcat}} za operaciju stepenovanja.%
\end{isamarkuptext}\isamarkuptrue%
\isacommand{fun}\isamarkupfalse%
\ stepenuj\ {\isacharparenleft}{\kern0pt}\isakeyword{infixr}\ {\isachardoublequoteopen}{\isasymZcat}{\isachardoublequoteclose}\ {\isadigit{1}}{\isadigit{0}}{\isadigit{2}}{\isacharparenright}{\kern0pt}\ \isakeyword{where}\isanewline
\ \ {\isachardoublequoteopen}a\ {\isasymZcat}\ b\ {\isacharequal}{\kern0pt}\ undefined{\isachardoublequoteclose}%
\begin{isamarkuptext}%
Pokazati da važi: $a^1 = a$.%
\end{isamarkuptext}\isamarkuptrue%
\isacommand{lemma}\isamarkupfalse%
\ stepenuj{\isacharunderscore}{\kern0pt}jedan{\isacharcolon}{\kern0pt}\isanewline
\ \ \isakeyword{shows}\ {\isachardoublequoteopen}a\ {\isasymZcat}\ {\isasymone}\ {\isacharequal}{\kern0pt}\ a{\isachardoublequoteclose}\isanewline
\ \ %
\isadelimproof
%
\endisadelimproof
%
\isatagproof
%
\endisatagproof
{\isafoldproof}%
%
\isadelimproof
%
\endisadelimproof
%
\begin{isamarkuptext}%
Pokazati da važi: $a^{(n+m)} = a^n b^m$.%
\end{isamarkuptext}\isamarkuptrue%
\isacommand{lemma}\isamarkupfalse%
\ stepenuj{\isacharunderscore}{\kern0pt}na{\isacharunderscore}{\kern0pt}zbir{\isacharbrackleft}{\kern0pt}simp{\isacharbrackright}{\kern0pt}{\isacharcolon}{\kern0pt}\isanewline
\ \ \isakeyword{shows}\ {\isachardoublequoteopen}a\ {\isasymZcat}\ {\isacharparenleft}{\kern0pt}n\ {\isasymoplus}\ m{\isacharparenright}{\kern0pt}\ {\isacharequal}{\kern0pt}\ a\ {\isasymZcat}\ n\ {\isasymotimes}\ a\ {\isasymZcat}\ m{\isachardoublequoteclose}\isanewline
\ \ %
\isadelimproof
%
\endisadelimproof
%
\isatagproof
%
\endisatagproof
{\isafoldproof}%
%
\isadelimproof
%
\endisadelimproof
%
\begin{isamarkuptext}%
Pokazati da važi: $a^{nm} = a^{n^m}$.%
\end{isamarkuptext}\isamarkuptrue%
\isacommand{lemma}\isamarkupfalse%
\ stepenuj{\isacharunderscore}{\kern0pt}na{\isacharunderscore}{\kern0pt}proizvod{\isacharcolon}{\kern0pt}\isanewline
\ \ \isakeyword{shows}\ {\isachardoublequoteopen}a\ {\isasymZcat}\ {\isacharparenleft}{\kern0pt}n\ {\isasymotimes}\ m{\isacharparenright}{\kern0pt}\ {\isacharequal}{\kern0pt}\ {\isacharparenleft}{\kern0pt}a\ {\isasymZcat}\ n{\isacharparenright}{\kern0pt}\ {\isasymZcat}\ m{\isachardoublequoteclose}\isanewline
\ \ %
\isadelimproof
%
\endisadelimproof
%
\isatagproof
%
\endisatagproof
{\isafoldproof}%
%
\isadelimproof
%
\endisadelimproof
%
\end{exercise}
%
\begin{exercise}[subtitle=Dodatni primeri.]
%
\begin{isamarkuptext}%
Pokazati sledeće teoreme u Isar-u. Kao dodatan izazov, dozvoljeno je korišćenje samo 
      primenjivanje pravila \isa{rule} i \isa{subst} za dokazivanje među koraka, tj. bilo
      kakva automatizacija (\isa{simp{\isacharcomma}{\kern0pt}\ auto{\isacharcomma}{\kern0pt}\ metis{\isacharcomma}{\kern0pt}\ blast{\isacharcomma}{\kern0pt}\ force{\isacharcomma}{\kern0pt}\ fastforce{\isacharcomma}{\kern0pt}\ sladgehammer{\isacharcomma}{\kern0pt}\ {\isachardot}{\kern0pt}{\isachardot}{\kern0pt}{\isachardot}{\kern0pt}}) 
      je zabranjena.%
\end{isamarkuptext}\isamarkuptrue%
\isacommand{lemma}\isamarkupfalse%
\ {\isachardoublequoteopen}a\ {\isasymoplus}\ Nula\ {\isacharequal}{\kern0pt}\ a{\isachardoublequoteclose}\isanewline
\ \ %
\isadelimproof
%
\endisadelimproof
%
\isatagproof
%
\endisatagproof
{\isafoldproof}%
%
\isadelimproof
%
\endisadelimproof
\isanewline
\isacommand{lemma}\isamarkupfalse%
\ {\isachardoublequoteopen}a\ {\isasymotimes}\ {\isacharparenleft}{\kern0pt}Sled\ b{\isacharparenright}{\kern0pt}\ {\isacharequal}{\kern0pt}\ a\ {\isasymotimes}\ b\ {\isasymoplus}\ a{\isachardoublequoteclose}\isanewline
\ \ %
\isadelimproof
%
\endisadelimproof
%
\isatagproof
%
\endisatagproof
{\isafoldproof}%
%
\isadelimproof
%
\endisadelimproof
\isanewline
\isacommand{lemma}\isamarkupfalse%
\ {\isachardoublequoteopen}a\ {\isasymotimes}\ b\ {\isasymotimes}\ c\ {\isacharequal}{\kern0pt}\ a\ {\isasymotimes}\ {\isacharparenleft}{\kern0pt}b\ {\isasymotimes}\ c{\isacharparenright}{\kern0pt}{\isachardoublequoteclose}\isanewline
\ \ %
\isadelimproof
%
\endisadelimproof
%
\isatagproof
%
\endisatagproof
{\isafoldproof}%
%
\isadelimproof
%
\endisadelimproof
\isanewline
\isacommand{lemma}\isamarkupfalse%
\ {\isachardoublequoteopen}a\ {\isasymotimes}\ b\ {\isacharequal}{\kern0pt}\ b\ {\isasymotimes}\ a{\isachardoublequoteclose}\isanewline
\ \ %
\isadelimproof
%
\endisadelimproof
%
\isatagproof
%
\endisatagproof
{\isafoldproof}%
%
\isadelimproof
%
\endisadelimproof
\isanewline
\isacommand{lemma}\isamarkupfalse%
\ {\isachardoublequoteopen}a\ {\isasymotimes}\ {\isacharparenleft}{\kern0pt}b\ {\isasymoplus}\ c{\isacharparenright}{\kern0pt}\ {\isacharequal}{\kern0pt}\ a\ {\isasymotimes}\ b\ {\isasymoplus}\ a\ {\isasymotimes}\ c{\isachardoublequoteclose}\isanewline
\ \ %
\isadelimproof
%
\endisadelimproof
%
\isatagproof
%
\endisatagproof
{\isafoldproof}%
%
\isadelimproof
%
\endisadelimproof
%
\end{exercise}
%
\isadelimtheory
%
\endisadelimtheory
%
\isatagtheory
%
\endisatagtheory
{\isafoldtheory}%
%
\isadelimtheory
%
\endisadelimtheory
%
\end{isabellebody}%
\endinput
%:%file=MyTheory.tex%:%
%:%18=8%:%
%:%21=10%:%
%:%23=12%:%
%:%24=12%:%
%:%26=14%:%
%:%28=16%:%
%:%29=16%:%
%:%30=17%:%
%:%31=18%:%
%:%32=18%:%
%:%33=19%:%
%:%34=19%:%
%:%35=20%:%
%:%36=20%:%
%:%38=23%:%
%:%40=25%:%
%:%41=25%:%
%:%42=26%:%
%:%43=27%:%
%:%44=28%:%
%:%45=28%:%
%:%46=29%:%
%:%47=30%:%
%:%48=31%:%
%:%49=31%:%
%:%50=32%:%
%:%52=34%:%
%:%53=35%:%
%:%55=37%:%
%:%56=37%:%
%:%57=38%:%
%:%59=40%:%
%:%63=42%:%
%:%65=44%:%
%:%66=44%:%
%:%67=45%:%
%:%68=46%:%
%:%83=48%:%
%:%84=49%:%
%:%86=51%:%
%:%87=51%:%
%:%88=52%:%
%:%89=53%:%
%:%102=54%:%
%:%103=55%:%
%:%104=55%:%
%:%105=56%:%
%:%106=57%:%
%:%121=59%:%
%:%122=60%:%
%:%124=62%:%
%:%125=62%:%
%:%126=63%:%
%:%128=65%:%
%:%129=66%:%
%:%131=68%:%
%:%132=68%:%
%:%133=69%:%
%:%134=70%:%
%:%149=72%:%
%:%151=74%:%
%:%152=74%:%
%:%153=75%:%
%:%154=76%:%
%:%169=78%:%
%:%170=79%:%
%:%172=81%:%
%:%173=81%:%
%:%174=82%:%
%:%176=84%:%
%:%178=86%:%
%:%179=86%:%
%:%180=87%:%
%:%181=88%:%
%:%196=90%:%
%:%198=92%:%
%:%199=92%:%
%:%200=93%:%
%:%201=94%:%
%:%216=96%:%
%:%218=98%:%
%:%219=98%:%
%:%220=99%:%
%:%221=100%:%
%:%235=102%:%
%:%237=104%:%
%:%240=106%:%
%:%241=107%:%
%:%242=108%:%
%:%243=109%:%
%:%245=111%:%
%:%246=111%:%
%:%247=112%:%
%:%260=113%:%
%:%261=114%:%
%:%262=114%:%
%:%263=115%:%
%:%276=116%:%
%:%277=117%:%
%:%278=117%:%
%:%279=118%:%
%:%292=119%:%
%:%293=120%:%
%:%294=120%:%
%:%295=121%:%
%:%308=122%:%
%:%309=123%:%
%:%310=123%:%
%:%311=124%:%
%:%325=126%:%
