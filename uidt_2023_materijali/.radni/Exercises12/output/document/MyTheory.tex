%
\begin{isabellebody}%
\setisabellecontext{MyTheory}%
%
\isadelimtheory
%
\endisadelimtheory
%
\isatagtheory
%
\endisatagtheory
{\isafoldtheory}%
%
\isadelimtheory
%
\endisadelimtheory
%
\begin{exercise}[subtitle=stek mašina.]
%
\begin{isamarkuptext}%
Definisati algebarski tip podataka \isa{izraz} koji predstavlja
      izraz koga čine konstante koje su prirodni brojevi, i tri
      binarne operacije plus, minus, i puta nad izrazom.%
\end{isamarkuptext}\isamarkuptrue%
%
\begin{isamarkuptext}%
Konstruisati proizvoljnu instancu tipa \isa{izraz} i proveriti
      njenu ispravnost pomoću ključne reči \isa{term}.%
\end{isamarkuptext}\isamarkuptrue%
%
\begin{isamarkuptext}%
Definisati funkciju \isa{vrednost\ {\isacharcolon}{\kern0pt}{\isacharcolon}{\kern0pt}\ izraz\ {\isasymRightarrow}\ nat} koja računa
      vrednost izraza.%
\end{isamarkuptext}\isamarkuptrue%
%
\begin{isamarkuptext}%
Definisati izraze \isa{x{\isadigit{1}}}, \isa{x{\isadigit{2}}}, i \isa{x{\isadigit{3}}}, gde je
      $x_1 \equiv 2 + 3$, $x_2 \equiv 3 \cdot (5 - 2)$, i $x_3 \equiv 3 \cdot 4 \cdot (5 - 2)$.
      Izračunati vrednosti ovih izraza.%
\end{isamarkuptext}\isamarkuptrue%
%
\begin{isamarkuptext}%
Definisati tip \isa{stek} kao listu prirodnih brojeva. 
      Dodavanje na vrh steka podrazumeva operaciju 
      \isa{Cons} (dodavanje na početak liste).%
\end{isamarkuptext}\isamarkuptrue%
%
\begin{isamarkuptext}%
Definisati algebarski tip \isa{operacija} koji predstavlja
      moguće operacije koje će se izvršavati nad stekom.
      Nad stekom je moguće primeniti operaciju za plus,
      minus, puta i dodavanje nogov elementa na stek.%
\end{isamarkuptext}\isamarkuptrue%
%
\begin{isamarkuptext}%
Definisati funkciju \isa{izvrsiOp\ {\isacharcolon}{\kern0pt}{\isacharcolon}{\kern0pt}\ operacija\ {\isasymRightarrow}\ stek\ {\isasymRightarrow}\ stek} koja 
      izvršava datu operaciju nad stekom i vraća novo stanje steka.%
\end{isamarkuptext}\isamarkuptrue%
%
\begin{isamarkuptext}%
Definisati tip \isa{program} kao listu operacija.%
\end{isamarkuptext}\isamarkuptrue%
%
\begin{isamarkuptext}%
Definisati funkciju \isa{prevedi\ {\isacharcolon}{\kern0pt}{\isacharcolon}{\kern0pt}\ izraz\ {\isasymRightarrow}\ program} koja
      data izraz prevodi u listu operacija, tj. program.
      Primeniti ovu funkciju nad izrazima \isa{x{\isadigit{1}}}, \isa{x{\isadigit{2}}}, i \isa{x{\isadigit{3}}}.%
\end{isamarkuptext}\isamarkuptrue%
%
\begin{isamarkuptext}%
Definisati funkciju \isa{izvrsiProgram\ {\isacharcolon}{\kern0pt}{\isacharcolon}{\kern0pt}\ program\ {\isasymRightarrow}\ stek\ {\isasymRightarrow}\ stek}
      koja primenjuje listu operacija, tj. program nad stekom.
      Izračunati vrednost ove funkcije kada se primeni nad
      programom (koji se dobiju prevođenjem izraza \isa{x{\isadigit{1}}}, \isa{x{\isadigit{2}}}, i \isa{x{\isadigit{3}}})
      i praznim stekom.%
\end{isamarkuptext}\isamarkuptrue%
%
\begin{isamarkuptext}%
Dodatno, definisati funkciju \isa{racunar\ {\isacharcolon}{\kern0pt}{\isacharcolon}{\kern0pt}\ izraz\ {\isasymRightarrow}\ nat} koja
      prevodi program izvršava program (koji se dobija prevođenjem izraza)
      nad praznim stekom. Takođe, testirati ovu funkciju nad izrazima
      \isa{x{\isadigit{1}}}, \isa{x{\isadigit{2}}}, i \isa{x{\isadigit{3}}}.%
\end{isamarkuptext}\isamarkuptrue%
%
\begin{isamarkuptext}%
Pokazati da računar korektno izračunava vrednost izraza, tj. da su
      funkcije \isa{racunar} i \isa{vrednost} ekvivalentne.
      \isa{Savet}: Potrebno je definisati pomoćne leme generalizacijom.%
\end{isamarkuptext}\isamarkuptrue%
%
\end{exercise}
%
\isadelimtheory
%
\endisadelimtheory
%
\isatagtheory
%
\endisatagtheory
{\isafoldtheory}%
%
\isadelimtheory
%
\endisadelimtheory
%
\end{isabellebody}%
\endinput
%:%file=MyTheory.tex%:%
%:%18=8%:%
%:%21=10%:%
%:%22=11%:%
%:%23=12%:%
%:%27=14%:%
%:%28=15%:%
%:%32=17%:%
%:%33=18%:%
%:%37=20%:%
%:%38=21%:%
%:%39=22%:%
%:%43=24%:%
%:%44=25%:%
%:%45=26%:%
%:%49=28%:%
%:%50=29%:%
%:%51=30%:%
%:%52=31%:%
%:%56=33%:%
%:%57=34%:%
%:%61=36%:%
%:%65=38%:%
%:%66=39%:%
%:%67=40%:%
%:%71=42%:%
%:%72=43%:%
%:%73=44%:%
%:%74=45%:%
%:%75=46%:%
%:%79=48%:%
%:%80=49%:%
%:%81=50%:%
%:%82=51%:%
%:%86=53%:%
%:%87=54%:%
%:%88=55%:%
%:%91=57%:%
