%
\begin{isabellebody}%
\setisabellecontext{MyTheory}%
%
\isadelimtheory
%
\endisadelimtheory
%
\isatagtheory
%
\endisatagtheory
{\isafoldtheory}%
%
\isadelimtheory
%
\endisadelimtheory
%
\begin{exercise}[subtitle=Intuicionistička pravila prirodne dedukcije u logici prvog reda]
%
\begin{isamarkuptext}%
Diskutovati o pravilima uvođenja i pravilima eliminacije prirodne dedukcije u logici 
      prvog reda. Pomoću ključne reči \isa{thm} ispitati svako pravilo prirodne dedukcije. 
      Primeniti odgovarajuće pravilo prirodne dedukcije na jednostavnim formulama i diskutovati 
      o cilju koga treba dokazati pre i posle primene tog pravila.%
\end{isamarkuptext}\isamarkuptrue%
%
\begin{isamarkuptext}%
Za logiku prvog reda pored pravila prirodne dedukcije iskazne 
      logike, važe i pravila uvođenja i elimenacije kvantifikatora.%
\end{isamarkuptext}\isamarkuptrue%
%
\begin{isamarkuptext}%
Uvođenje univerzalnog kvantifikatora: \isa{allI}%
\end{isamarkuptext}\isamarkuptrue%
\isacommand{lemma}\isamarkupfalse%
\ {\isachardoublequoteopen}{\isasymforall}\ x{\isachardot}{\kern0pt}\ P\ x{\isachardoublequoteclose}\isanewline
\ \ %
\isadelimproof
%
\endisadelimproof
%
\isatagproof
%
\endisatagproof
{\isafoldproof}%
%
\isadelimproof
%
\endisadelimproof
%
\begin{isamarkuptext}%
Eliminacija univerzalnog kvantifikatora: \isa{allE}%
\end{isamarkuptext}\isamarkuptrue%
\isacommand{lemma}\isamarkupfalse%
\ {\isachardoublequoteopen}{\isasymforall}\ x{\isachardot}{\kern0pt}\ P\ x\ {\isasymLongrightarrow}\ A{\isachardoublequoteclose}\isanewline
\ \ %
\isadelimproof
%
\endisadelimproof
%
\isatagproof
%
\endisatagproof
{\isafoldproof}%
%
\isadelimproof
%
\endisadelimproof
%
\begin{isamarkuptext}%
Uvođenje egzistencijalnog kvantifikatora: \isa{exI}%
\end{isamarkuptext}\isamarkuptrue%
\isacommand{lemma}\isamarkupfalse%
\ {\isachardoublequoteopen}{\isasymexists}\ x{\isachardot}{\kern0pt}\ P\ x{\isachardoublequoteclose}\isanewline
\ \ %
\isadelimproof
%
\endisadelimproof
%
\isatagproof
%
\endisatagproof
{\isafoldproof}%
%
\isadelimproof
%
\endisadelimproof
%
\begin{isamarkuptext}%
Eliminacija egzistencijalnog kvantifikatora: \isa{exE}%
\end{isamarkuptext}\isamarkuptrue%
\isacommand{lemma}\isamarkupfalse%
\ {\isachardoublequoteopen}{\isasymexists}\ x{\isachardot}{\kern0pt}\ P\ x\ {\isasymLongrightarrow}\ A{\isachardoublequoteclose}\isanewline
\ \ %
\isadelimproof
%
\endisadelimproof
%
\isatagproof
%
\endisatagproof
{\isafoldproof}%
%
\isadelimproof
%
\endisadelimproof
%
\end{exercise}
%
\begin{exercise}[subtitle=Dokazi u prirodnoj dedukciji]
%
\begin{isamarkuptext}%
Pokazati da su sledeće formule valjane u logici prvog reda. 
      Dozvoljeno je korišćenje samo intuicionističkih pravila prirodne dedukcije.%
\end{isamarkuptext}\isamarkuptrue%
\isacommand{lemma}\isamarkupfalse%
\ {\isachardoublequoteopen}{\isacharparenleft}{\kern0pt}{\isasymforall}\ x{\isachardot}{\kern0pt}\ Man\ x\ {\isasymlongrightarrow}\ Mortal\ x{\isacharparenright}{\kern0pt}\ {\isasymand}\ Man\ Socrates\ {\isasymlongrightarrow}\ Mortal\ Socrates{\isachardoublequoteclose}\isanewline
\ \ %
\isadelimproof
%
\endisadelimproof
%
\isatagproof
%
\endisatagproof
{\isafoldproof}%
%
\isadelimproof
%
\endisadelimproof
\isanewline
\isacommand{lemma}\isamarkupfalse%
\ de{\isacharunderscore}{\kern0pt}Morgan{\isacharunderscore}{\kern0pt}{\isadigit{1}}{\isacharcolon}{\kern0pt}\ {\isachardoublequoteopen}{\isacharparenleft}{\kern0pt}{\isasymexists}\ x{\isachardot}{\kern0pt}\ {\isasymnot}\ P\ x{\isacharparenright}{\kern0pt}\ {\isasymlongrightarrow}\ {\isasymnot}\ {\isacharparenleft}{\kern0pt}{\isasymforall}\ x{\isachardot}{\kern0pt}\ P\ x{\isacharparenright}{\kern0pt}{\isachardoublequoteclose}\isanewline
\ \ %
\isadelimproof
%
\endisadelimproof
%
\isatagproof
%
\endisatagproof
{\isafoldproof}%
%
\isadelimproof
%
\endisadelimproof
\isanewline
\isacommand{lemma}\isamarkupfalse%
\ de{\isacharunderscore}{\kern0pt}Morgan{\isacharunderscore}{\kern0pt}{\isadigit{2}}{\isacharcolon}{\kern0pt}\ {\isachardoublequoteopen}{\isacharparenleft}{\kern0pt}{\isasymforall}\ x{\isachardot}{\kern0pt}\ {\isasymnot}\ P\ x{\isacharparenright}{\kern0pt}\ {\isasymlongrightarrow}\ {\isacharparenleft}{\kern0pt}{\isasymnexists}\ x{\isachardot}{\kern0pt}\ P\ x{\isacharparenright}{\kern0pt}{\isachardoublequoteclose}\isanewline
\ \ %
\isadelimproof
%
\endisadelimproof
%
\isatagproof
%
\endisatagproof
{\isafoldproof}%
%
\isadelimproof
%
\endisadelimproof
\isanewline
\isacommand{lemma}\isamarkupfalse%
\ de{\isacharunderscore}{\kern0pt}Morgan{\isacharunderscore}{\kern0pt}{\isadigit{3}}{\isacharcolon}{\kern0pt}\ {\isachardoublequoteopen}{\isacharparenleft}{\kern0pt}{\isasymnexists}\ x{\isachardot}{\kern0pt}\ P\ x{\isacharparenright}{\kern0pt}\ {\isasymlongrightarrow}\ {\isacharparenleft}{\kern0pt}{\isasymforall}\ x{\isachardot}{\kern0pt}\ {\isasymnot}\ P\ x{\isacharparenright}{\kern0pt}{\isachardoublequoteclose}\isanewline
\ \ %
\isadelimproof
%
\endisadelimproof
%
\isatagproof
%
\endisatagproof
{\isafoldproof}%
%
\isadelimproof
%
\endisadelimproof
\isanewline
\isacommand{lemma}\isamarkupfalse%
\ {\isachardoublequoteopen}{\isacharparenleft}{\kern0pt}{\isasymexists}\ x{\isachardot}{\kern0pt}\ P\ x{\isacharparenright}{\kern0pt}\ {\isasymand}\ {\isacharparenleft}{\kern0pt}{\isasymforall}\ x{\isachardot}{\kern0pt}\ P\ x\ {\isasymlongrightarrow}\ Q\ x{\isacharparenright}{\kern0pt}\ {\isasymlongrightarrow}\ {\isacharparenleft}{\kern0pt}{\isasymexists}\ x{\isachardot}{\kern0pt}\ Q\ x{\isacharparenright}{\kern0pt}{\isachardoublequoteclose}\isanewline
\ \ %
\isadelimproof
%
\endisadelimproof
%
\isatagproof
%
\endisatagproof
{\isafoldproof}%
%
\isadelimproof
%
\endisadelimproof
\isanewline
\isacommand{lemma}\isamarkupfalse%
\ {\isachardoublequoteopen}{\isacharparenleft}{\kern0pt}{\isasymforall}\ m{\isachardot}{\kern0pt}\ Man\ m\ {\isasymlongrightarrow}\ Mortal\ m{\isacharparenright}{\kern0pt}\ {\isasymand}\ \isanewline
\ \ \ \ \ \ \ {\isacharparenleft}{\kern0pt}{\isasymforall}\ g{\isachardot}{\kern0pt}\ Greek\ g\ {\isasymlongrightarrow}\ Man\ g{\isacharparenright}{\kern0pt}\ {\isasymlongrightarrow}\isanewline
\ \ \ \ \ \ \ {\isacharparenleft}{\kern0pt}{\isasymforall}\ a{\isachardot}{\kern0pt}\ Greek\ a\ {\isasymlongrightarrow}\ Mortal\ a{\isacharparenright}{\kern0pt}{\isachardoublequoteclose}\isanewline
\ \ %
\isadelimproof
%
\endisadelimproof
%
\isatagproof
%
\endisatagproof
{\isafoldproof}%
%
\isadelimproof
%
\endisadelimproof
%
\begin{isamarkuptext}%
Dodatni primeri:%
\end{isamarkuptext}\isamarkuptrue%
\isacommand{lemma}\isamarkupfalse%
\ {\isachardoublequoteopen}{\isacharparenleft}{\kern0pt}{\isasymforall}\ a{\isachardot}{\kern0pt}\ P\ a\ {\isasymlongrightarrow}\ Q\ a{\isacharparenright}{\kern0pt}\ {\isasymand}\ {\isacharparenleft}{\kern0pt}{\isasymforall}\ b{\isachardot}{\kern0pt}\ P\ b{\isacharparenright}{\kern0pt}\ {\isasymlongrightarrow}\ {\isacharparenleft}{\kern0pt}{\isasymforall}\ x{\isachardot}{\kern0pt}\ Q\ x{\isacharparenright}{\kern0pt}{\isachardoublequoteclose}\isanewline
\ \ %
\isadelimproof
%
\endisadelimproof
%
\isatagproof
%
\endisatagproof
{\isafoldproof}%
%
\isadelimproof
%
\endisadelimproof
\isanewline
\isacommand{lemma}\isamarkupfalse%
\ {\isachardoublequoteopen}{\isacharparenleft}{\kern0pt}{\isasymexists}\ x{\isachardot}{\kern0pt}\ A\ x\ {\isasymor}\ B\ x{\isacharparenright}{\kern0pt}\ {\isasymlongrightarrow}\ {\isacharparenleft}{\kern0pt}{\isasymexists}\ x{\isachardot}{\kern0pt}\ A\ x{\isacharparenright}{\kern0pt}\ {\isasymor}\ {\isacharparenleft}{\kern0pt}{\isasymexists}\ x{\isachardot}{\kern0pt}\ B\ x{\isacharparenright}{\kern0pt}{\isachardoublequoteclose}\isanewline
\ \ %
\isadelimproof
%
\endisadelimproof
%
\isatagproof
%
\endisatagproof
{\isafoldproof}%
%
\isadelimproof
%
\endisadelimproof
\isanewline
\isacommand{lemma}\isamarkupfalse%
\ {\isachardoublequoteopen}{\isacharparenleft}{\kern0pt}{\isasymforall}\ x{\isachardot}{\kern0pt}\ A\ x\ {\isasymlongrightarrow}\ {\isasymnot}\ B\ x{\isacharparenright}{\kern0pt}\ {\isasymlongrightarrow}\ {\isacharparenleft}{\kern0pt}{\isasymnexists}\ x{\isachardot}{\kern0pt}\ A\ x\ {\isasymand}\ B\ x{\isacharparenright}{\kern0pt}{\isachardoublequoteclose}\isanewline
\ \ %
\isadelimproof
%
\endisadelimproof
%
\isatagproof
%
\endisatagproof
{\isafoldproof}%
%
\isadelimproof
%
\endisadelimproof
%
\begin{isamarkuptext}%
Formulisati i dokazati naredna tvrđenja.%
\end{isamarkuptext}\isamarkuptrue%
%
\begin{isamarkuptext}%
Ako za svaki broj koji nije paran važi da je neparan;\\
      i ako za svaki neparan broj važi da nije paran;\\
      pokazati da onda za svaki broj važi da nije istovremeno i paran i neparan.%
\end{isamarkuptext}\isamarkuptrue%
%
\begin{isamarkuptext}%
Ako je svaki kvadrat romb;\\
      i ako je svaki kvadrat pravougaonik;\\
      i ako znamo da postoji makar jedan kvadrat;\\
      onda postoji makar jedan romb koji je istovremeno i pravougaonik.%
\end{isamarkuptext}\isamarkuptrue%
%
\begin{isamarkuptext}%
Ako je relacija R simetrična, tranzitivna\\
      i ako za svako x postoji y koje je sa njim u relaciji,\\ 
      onda je relacija R i refleksivna.%
\end{isamarkuptext}\isamarkuptrue%
%
\begin{isamarkuptext}%
Savet: Pomoću ključne reči \isa{definition} definisati osobinu refleksivnosti,
      tranzitivnosti i simetricnosti. Ta formulisati tvđenje i dokazati ga.
      Podsetiti se ključne reči \isa{unfolding} za raspisivanje definicije.%
\end{isamarkuptext}\isamarkuptrue%
%
\end{exercise}
%
\begin{exercise}[subtitle=Klasična pravilo prirodne dedukcije: ccontr.]
%
\begin{isamarkuptext}%
Diskutovati zašto sledeće tvrđenje može biti dokazano samo intuicionističkim pravilima 
      prirodne dedukcije, dok to ne važi za tvrđenje nakon njega. Primetiti razliku između 
      pravila \isa{notI} i \isa{ccontr}.%
\end{isamarkuptext}\isamarkuptrue%
\isacommand{lemma}\isamarkupfalse%
\ {\isacartoucheopen}A\ {\isasymlongrightarrow}\ {\isasymnot}\ {\isasymnot}\ A{\isacartoucheclose}\isanewline
\ \ %
\isadelimproof
%
\endisadelimproof
%
\isatagproof
%
\endisatagproof
{\isafoldproof}%
%
\isadelimproof
%
\endisadelimproof
\isanewline
\isacommand{lemma}\isamarkupfalse%
\ {\isachardoublequoteopen}{\isasymnot}\ {\isasymnot}\ A\ {\isasymlongrightarrow}\ A{\isachardoublequoteclose}\isanewline
\ \ %
\isadelimproof
%
\endisadelimproof
%
\isatagproof
%
\endisatagproof
{\isafoldproof}%
%
\isadelimproof
%
\endisadelimproof
%
\begin{isamarkuptext}%
Dokazati sledeća tvrđenja:%
\end{isamarkuptext}\isamarkuptrue%
\isacommand{lemma}\isamarkupfalse%
\ {\isachardoublequoteopen}{\isacharparenleft}{\kern0pt}{\isasymnot}\ P\ {\isasymlongrightarrow}\ P{\isacharparenright}{\kern0pt}\ {\isasymlongrightarrow}\ P{\isachardoublequoteclose}\isanewline
\ \ %
\isadelimproof
%
\endisadelimproof
%
\isatagproof
%
\endisatagproof
{\isafoldproof}%
%
\isadelimproof
%
\endisadelimproof
\isanewline
\isacommand{lemma}\isamarkupfalse%
\ {\isachardoublequoteopen}{\isasymnot}\ {\isacharparenleft}{\kern0pt}A\ {\isasymand}\ B{\isacharparenright}{\kern0pt}\ {\isasymlongrightarrow}\ {\isasymnot}\ A\ {\isasymor}\ {\isasymnot}\ B{\isachardoublequoteclose}\isanewline
\ \ %
\isadelimproof
%
\endisadelimproof
%
\isatagproof
%
\endisatagproof
{\isafoldproof}%
%
\isadelimproof
%
\endisadelimproof
\isanewline
\isacommand{lemma}\isamarkupfalse%
\ {\isachardoublequoteopen}{\isacharparenleft}{\kern0pt}{\isasymnot}\ {\isacharparenleft}{\kern0pt}{\isasymforall}\ x{\isachardot}{\kern0pt}\ P\ x{\isacharparenright}{\kern0pt}{\isacharparenright}{\kern0pt}\ {\isasymlongrightarrow}\ {\isacharparenleft}{\kern0pt}{\isasymexists}\ x{\isachardot}{\kern0pt}\ {\isasymnot}\ P\ x{\isacharparenright}{\kern0pt}{\isachardoublequoteclose}\isanewline
\ \ %
\isadelimproof
%
\endisadelimproof
%
\isatagproof
%
\endisatagproof
{\isafoldproof}%
%
\isadelimproof
%
\endisadelimproof
%
\begin{isamarkuptext}%
Dodatni primeri:%
\end{isamarkuptext}\isamarkuptrue%
\isacommand{lemma}\isamarkupfalse%
\ {\isachardoublequoteopen}{\isacharparenleft}{\kern0pt}{\isasymnot}\ B\ {\isasymlongrightarrow}\ {\isasymnot}\ A{\isacharparenright}{\kern0pt}\ {\isasymlongrightarrow}\ {\isacharparenleft}{\kern0pt}A\ {\isasymlongrightarrow}\ B{\isacharparenright}{\kern0pt}{\isachardoublequoteclose}\isanewline
\ \ %
\isadelimproof
%
\endisadelimproof
%
\isatagproof
%
\endisatagproof
{\isafoldproof}%
%
\isadelimproof
%
\endisadelimproof
\isanewline
\isacommand{lemma}\isamarkupfalse%
\ {\isachardoublequoteopen}{\isacharparenleft}{\kern0pt}A\ {\isasymlongrightarrow}\ B{\isacharparenright}{\kern0pt}\ {\isasymlongrightarrow}\ {\isacharparenleft}{\kern0pt}{\isasymnot}\ A\ {\isasymor}\ B{\isacharparenright}{\kern0pt}{\isachardoublequoteclose}\isanewline
\ \ %
\isadelimproof
%
\endisadelimproof
%
\isatagproof
%
\endisatagproof
{\isafoldproof}%
%
\isadelimproof
%
\endisadelimproof
\isanewline
\isacommand{lemma}\isamarkupfalse%
\ {\isachardoublequoteopen}{\isacharparenleft}{\kern0pt}{\isasymnot}\ P\ {\isasymlongrightarrow}\ Q{\isacharparenright}{\kern0pt}\ {\isasymlongleftrightarrow}\ {\isacharparenleft}{\kern0pt}{\isasymnot}\ Q\ {\isasymlongrightarrow}\ P{\isacharparenright}{\kern0pt}{\isachardoublequoteclose}\isanewline
\ \ %
\isadelimproof
%
\endisadelimproof
%
\isatagproof
%
\endisatagproof
{\isafoldproof}%
%
\isadelimproof
%
\endisadelimproof
\isanewline
\isacommand{lemma}\isamarkupfalse%
\ {\isachardoublequoteopen}{\isacharparenleft}{\kern0pt}{\isacharparenleft}{\kern0pt}P\ {\isasymlongrightarrow}\ Q{\isacharparenright}{\kern0pt}\ {\isasymlongrightarrow}\ P{\isacharparenright}{\kern0pt}\ {\isasymlongrightarrow}\ P{\isachardoublequoteclose}\isanewline
\ \ %
\isadelimproof
%
\endisadelimproof
%
\isatagproof
%
\endisatagproof
{\isafoldproof}%
%
\isadelimproof
%
\endisadelimproof
%
\end{exercise}
%
\begin{exercise}[subtitle=Klasična pravilo prirodne dedukcije: classical.]
%
\begin{isamarkuptext}%
Pokazati naredna tvrđenja pomoću pravila \isa{classical}. Zgodna alternativa ovog pravila je 
      razdvajanje na slučajeve neke podformule.%
\end{isamarkuptext}\isamarkuptrue%
\isacommand{thm}\isamarkupfalse%
\ classical\isanewline
\isanewline
\isacommand{lemma}\isamarkupfalse%
\ {\isachardoublequoteopen}P\ {\isasymor}\ {\isasymnot}\ P{\isachardoublequoteclose}\isanewline
\ \ %
\isadelimproof
%
\endisadelimproof
%
\isatagproof
%
\endisatagproof
{\isafoldproof}%
%
\isadelimproof
%
\endisadelimproof
\isanewline
\isacommand{lemma}\isamarkupfalse%
\ {\isachardoublequoteopen}{\isacharparenleft}{\kern0pt}A\ {\isasymlongleftrightarrow}\ {\isacharparenleft}{\kern0pt}A\ {\isasymlongleftrightarrow}\ B{\isacharparenright}{\kern0pt}{\isacharparenright}{\kern0pt}\ {\isasymlongrightarrow}\ B{\isachardoublequoteclose}\isanewline
\ \ %
\isadelimproof
%
\endisadelimproof
%
\isatagproof
%
\endisatagproof
{\isafoldproof}%
%
\isadelimproof
%
\endisadelimproof
%
\begin{isamarkuptext}%
\isa{Paradoks\ pijanca}:\\
      Postoji osoba za koju važi, ako je on pijanac onda su i svi ostali pijanci.%
\end{isamarkuptext}\isamarkuptrue%
\isacommand{lemma}\isamarkupfalse%
\ drinker{\isacharprime}{\kern0pt}s{\isacharunderscore}{\kern0pt}paradox{\isacharcolon}{\kern0pt}\ {\isachardoublequoteopen}{\isasymexists}\ x{\isachardot}{\kern0pt}\ drunk\ x\ {\isasymlongrightarrow}\ {\isacharparenleft}{\kern0pt}{\isasymforall}\ x{\isachardot}{\kern0pt}\ drunk\ x{\isacharparenright}{\kern0pt}{\isachardoublequoteclose}\isanewline
\ \ %
\isadelimproof
%
\endisadelimproof
%
\isatagproof
%
\endisatagproof
{\isafoldproof}%
%
\isadelimproof
%
\endisadelimproof
%
\end{exercise}
%
\isadelimtheory
%
\endisadelimtheory
%
\isatagtheory
%
\endisatagtheory
{\isafoldtheory}%
%
\isadelimtheory
%
\endisadelimtheory
%
\end{isabellebody}%
\endinput
%:%file=MyTheory.tex%:%
%:%18=8%:%
%:%21=10%:%
%:%22=11%:%
%:%23=12%:%
%:%24=13%:%
%:%28=15%:%
%:%29=16%:%
%:%33=18%:%
%:%35=20%:%
%:%36=20%:%
%:%37=21%:%
%:%52=23%:%
%:%54=25%:%
%:%55=25%:%
%:%56=26%:%
%:%71=28%:%
%:%73=30%:%
%:%74=30%:%
%:%75=31%:%
%:%90=33%:%
%:%92=35%:%
%:%93=35%:%
%:%94=36%:%
%:%108=38%:%
%:%110=40%:%
%:%113=42%:%
%:%114=43%:%
%:%116=45%:%
%:%117=45%:%
%:%118=46%:%
%:%131=47%:%
%:%132=48%:%
%:%133=48%:%
%:%134=49%:%
%:%147=50%:%
%:%148=51%:%
%:%149=51%:%
%:%150=52%:%
%:%163=53%:%
%:%164=54%:%
%:%165=54%:%
%:%166=55%:%
%:%179=56%:%
%:%180=57%:%
%:%181=57%:%
%:%182=58%:%
%:%195=59%:%
%:%196=60%:%
%:%197=60%:%
%:%199=62%:%
%:%200=63%:%
%:%215=65%:%
%:%217=67%:%
%:%218=67%:%
%:%219=68%:%
%:%232=69%:%
%:%233=70%:%
%:%234=70%:%
%:%235=71%:%
%:%248=72%:%
%:%249=73%:%
%:%250=73%:%
%:%251=74%:%
%:%266=76%:%
%:%270=78%:%
%:%271=79%:%
%:%272=80%:%
%:%276=82%:%
%:%277=83%:%
%:%278=84%:%
%:%279=85%:%
%:%283=87%:%
%:%284=88%:%
%:%285=89%:%
%:%289=91%:%
%:%290=92%:%
%:%291=93%:%
%:%294=95%:%
%:%296=97%:%
%:%299=99%:%
%:%300=100%:%
%:%301=101%:%
%:%303=103%:%
%:%304=103%:%
%:%305=104%:%
%:%318=105%:%
%:%319=106%:%
%:%320=106%:%
%:%321=107%:%
%:%336=109%:%
%:%338=111%:%
%:%339=111%:%
%:%340=112%:%
%:%353=113%:%
%:%354=114%:%
%:%355=114%:%
%:%356=115%:%
%:%369=116%:%
%:%370=117%:%
%:%371=117%:%
%:%372=118%:%
%:%387=120%:%
%:%389=122%:%
%:%390=122%:%
%:%391=123%:%
%:%404=124%:%
%:%405=125%:%
%:%406=125%:%
%:%407=126%:%
%:%420=127%:%
%:%421=128%:%
%:%422=128%:%
%:%423=129%:%
%:%436=130%:%
%:%437=131%:%
%:%438=131%:%
%:%439=132%:%
%:%453=134%:%
%:%455=136%:%
%:%458=138%:%
%:%459=139%:%
%:%461=141%:%
%:%462=141%:%
%:%463=142%:%
%:%464=143%:%
%:%465=143%:%
%:%466=144%:%
%:%479=145%:%
%:%480=146%:%
%:%481=146%:%
%:%482=147%:%
%:%497=149%:%
%:%498=150%:%
%:%500=152%:%
%:%501=152%:%
%:%502=153%:%
%:%516=155%:%
