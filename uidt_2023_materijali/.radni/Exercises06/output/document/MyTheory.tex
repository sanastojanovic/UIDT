%
\begin{isabellebody}%
\setisabellecontext{MyTheory}%
%
\isadelimtheory
%
\endisadelimtheory
%
\isatagtheory
%
\endisatagtheory
{\isafoldtheory}%
%
\isadelimtheory
%
\endisadelimtheory
%
\begin{exercise}[subtitle=Svojstva funkcija]
%
\begin{isamarkuptext}%
Pokazati da je slika unije, unija pojedinačnih slika.\\
      \isa{Savet}: Razmotriti teoreme \isa{image{\isacharunderscore}{\kern0pt}def} i \isa{vimage{\isacharunderscore}{\kern0pt}def}.%
\end{isamarkuptext}\isamarkuptrue%
\isacommand{lemma}\isamarkupfalse%
\ image{\isacharunderscore}{\kern0pt}union{\isacharcolon}{\kern0pt}\isanewline
\ \ \isakeyword{shows}\ {\isachardoublequoteopen}f\ {\isacharbackquote}{\kern0pt}\ {\isacharparenleft}{\kern0pt}A\ {\isasymunion}\ B{\isacharparenright}{\kern0pt}\ {\isacharequal}{\kern0pt}\ f\ {\isacharbackquote}{\kern0pt}\ A\ {\isasymunion}\ f\ {\isacharbackquote}{\kern0pt}\ B{\isachardoublequoteclose}%
\isadelimproof
%
\endisadelimproof
%
\isatagproof
%
\endisatagproof
{\isafoldproof}%
%
\isadelimproof
%
\endisadelimproof
%
\begin{isamarkuptext}%
Neka je funkcija \isa{f} injektivna. 
      Pokazati da je slika preseka, presek pojedinačnih slika.\\
      \isa{Savet}: Razmotriti teoremu \isa{inj{\isacharunderscore}{\kern0pt}def}.%
\end{isamarkuptext}\isamarkuptrue%
\isacommand{lemma}\isamarkupfalse%
\ image{\isacharunderscore}{\kern0pt}inter{\isacharcolon}{\kern0pt}\ \isanewline
\ \ \isakeyword{assumes}\ {\isachardoublequoteopen}inj\ f{\isachardoublequoteclose}\isanewline
\ \ \ \ \isakeyword{shows}\ {\isachardoublequoteopen}f\ {\isacharbackquote}{\kern0pt}\ {\isacharparenleft}{\kern0pt}A\ {\isasyminter}\ B{\isacharparenright}{\kern0pt}\ {\isacharequal}{\kern0pt}\ f\ {\isacharbackquote}{\kern0pt}\ A\ {\isasyminter}\ f\ {\isacharbackquote}{\kern0pt}\ B{\isachardoublequoteclose}%
\isadelimproof
%
\endisadelimproof
%
\isatagproof
%
\endisatagproof
{\isafoldproof}%
%
\isadelimproof
%
\endisadelimproof
%
\begin{isamarkuptext}%
\isa{Savet}: Razmotriti teoremu \isa{surj{\isacharunderscore}{\kern0pt}def} i \isa{surjD}.%
\end{isamarkuptext}\isamarkuptrue%
\isacommand{lemma}\isamarkupfalse%
\ surj{\isacharunderscore}{\kern0pt}image{\isacharunderscore}{\kern0pt}vimage{\isacharcolon}{\kern0pt}\isanewline
\ \ \isakeyword{assumes}\ {\isachardoublequoteopen}surj\ f{\isachardoublequoteclose}\isanewline
\ \ \ \ \isakeyword{shows}\ {\isachardoublequoteopen}f\ {\isacharbackquote}{\kern0pt}\ {\isacharparenleft}{\kern0pt}f\ {\isacharminus}{\kern0pt}{\isacharbackquote}{\kern0pt}\ B{\isacharparenright}{\kern0pt}\ {\isacharequal}{\kern0pt}\ B{\isachardoublequoteclose}%
\isadelimproof
%
\endisadelimproof
%
\isatagproof
%
\endisatagproof
{\isafoldproof}%
%
\isadelimproof
%
\endisadelimproof
%
\begin{isamarkuptext}%
Pokazati da je kompozicija injektivna 
      ako su pojedinačne funkcije injektivne.\\
      \isa{Savet}: Razmotrite teoremu \isa{inj{\isacharunderscore}{\kern0pt}eq}.%
\end{isamarkuptext}\isamarkuptrue%
\isacommand{lemma}\isamarkupfalse%
\ comp{\isacharunderscore}{\kern0pt}inj{\isacharcolon}{\kern0pt}\isanewline
\ \ \isakeyword{assumes}\ {\isachardoublequoteopen}inj\ f{\isachardoublequoteclose}\isanewline
\ \ \ \ \ \ \isakeyword{and}\ {\isachardoublequoteopen}inj\ g{\isachardoublequoteclose}\isanewline
\ \ \ \ \isakeyword{shows}\ {\isachardoublequoteopen}inj\ {\isacharparenleft}{\kern0pt}f\ {\isasymcirc}\ g{\isacharparenright}{\kern0pt}{\isachardoublequoteclose}%
\isadelimproof
%
\endisadelimproof
%
\isatagproof
%
\endisatagproof
{\isafoldproof}%
%
\isadelimproof
%
\endisadelimproof
\isanewline
\isacommand{lemma}\isamarkupfalse%
\isanewline
\ \ \isakeyword{assumes}\ {\isachardoublequoteopen}inj\ f{\isachardoublequoteclose}\isanewline
\ \ \ \ \isakeyword{shows}\ {\isachardoublequoteopen}x\ {\isasymin}\ A\ {\isasymlongleftrightarrow}\ f\ x\ {\isasymin}\ f\ {\isacharbackquote}{\kern0pt}\ A{\isachardoublequoteclose}%
\isadelimproof
%
\endisadelimproof
%
\isatagproof
%
\endisatagproof
{\isafoldproof}%
%
\isadelimproof
%
\endisadelimproof
\isanewline
\isanewline
\isacommand{lemma}\isamarkupfalse%
\ inj{\isacharunderscore}{\kern0pt}vimage{\isacharunderscore}{\kern0pt}image{\isacharcolon}{\kern0pt}\isanewline
\ \ \isakeyword{assumes}\ {\isachardoublequoteopen}inj\ f{\isachardoublequoteclose}\isanewline
\ \ \ \ \isakeyword{shows}\ {\isachardoublequoteopen}f\ {\isacharminus}{\kern0pt}{\isacharbackquote}{\kern0pt}\ {\isacharparenleft}{\kern0pt}f\ {\isacharbackquote}{\kern0pt}\ A{\isacharparenright}{\kern0pt}\ {\isacharequal}{\kern0pt}\ A{\isachardoublequoteclose}%
\isadelimproof
%
\endisadelimproof
%
\isatagproof
%
\endisatagproof
{\isafoldproof}%
%
\isadelimproof
%
\endisadelimproof
%
\begin{isamarkuptext}%
Kompozicija je surjekcija
      ako su pojedinačne funkcije surjekcije.%
\end{isamarkuptext}\isamarkuptrue%
\isacommand{lemma}\isamarkupfalse%
\ comp{\isacharunderscore}{\kern0pt}surj{\isacharcolon}{\kern0pt}\isanewline
\ \ \isakeyword{assumes}\ {\isachardoublequoteopen}surj\ f{\isachardoublequoteclose}\isanewline
\ \ \ \ \ \ \isakeyword{and}\ {\isachardoublequoteopen}surj\ g{\isachardoublequoteclose}\isanewline
\ \ \ \ \isakeyword{shows}\ {\isachardoublequoteopen}surj\ {\isacharparenleft}{\kern0pt}f\ {\isasymcirc}\ g{\isacharparenright}{\kern0pt}{\isachardoublequoteclose}%
\isadelimproof
%
\endisadelimproof
%
\isatagproof
%
\endisatagproof
{\isafoldproof}%
%
\isadelimproof
%
\endisadelimproof
\isanewline
\isacommand{lemma}\isamarkupfalse%
\ vimage{\isacharunderscore}{\kern0pt}compl{\isacharcolon}{\kern0pt}\ \isanewline
\ \ \isakeyword{shows}\ {\isachardoublequoteopen}f\ {\isacharminus}{\kern0pt}{\isacharbackquote}{\kern0pt}\ {\isacharparenleft}{\kern0pt}{\isacharminus}{\kern0pt}\ B{\isacharparenright}{\kern0pt}\ {\isacharequal}{\kern0pt}\ {\isacharminus}{\kern0pt}\ {\isacharparenleft}{\kern0pt}f\ {\isacharminus}{\kern0pt}{\isacharbackquote}{\kern0pt}\ B{\isacharparenright}{\kern0pt}{\isachardoublequoteclose}%
\isadelimproof
%
\endisadelimproof
%
\isatagproof
%
\endisatagproof
{\isafoldproof}%
%
\isadelimproof
%
\endisadelimproof
%
\end{exercise}
%
\isadelimtheory
%
\endisadelimtheory
%
\isatagtheory
%
\endisatagtheory
{\isafoldtheory}%
%
\isadelimtheory
%
\endisadelimtheory
%
\end{isabellebody}%
\endinput
%:%file=MyTheory.tex%:%
%:%18=8%:%
%:%21=10%:%
%:%22=11%:%
%:%24=13%:%
%:%25=13%:%
%:%26=14%:%
%:%41=17%:%
%:%42=18%:%
%:%43=19%:%
%:%45=21%:%
%:%46=21%:%
%:%47=22%:%
%:%48=23%:%
%:%63=26%:%
%:%65=28%:%
%:%66=28%:%
%:%67=29%:%
%:%68=30%:%
%:%83=33%:%
%:%84=34%:%
%:%85=35%:%
%:%87=37%:%
%:%88=37%:%
%:%89=38%:%
%:%90=39%:%
%:%91=40%:%
%:%104=42%:%
%:%105=43%:%
%:%106=43%:%
%:%107=44%:%
%:%108=45%:%
%:%121=47%:%
%:%122=48%:%
%:%123=49%:%
%:%124=49%:%
%:%125=50%:%
%:%126=51%:%
%:%141=54%:%
%:%142=55%:%
%:%144=57%:%
%:%145=57%:%
%:%146=58%:%
%:%147=59%:%
%:%148=60%:%
%:%161=62%:%
%:%162=63%:%
%:%163=63%:%
%:%164=64%:%
%:%178=67%:%
